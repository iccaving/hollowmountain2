\section{Mig colloquialisms}

The following is a necessary update on the Migovec vernacular first published in the Hollow Mountain \citep{hm1}. This particular dialect of caving linguo is alive and thriving. 



\begin{description}
\item[Cow] Dried milk powder.
\item [Clag] What clouds look from the inside when enveloping the moutain
\item [Comf] Soft comfortable material to sit or lie on e.g. squares of chopped karrimats
\item [McGowan] A  dwarf pine based sofa so called because it resembles a bodybag containing the eponymous caver
\item [Faff]  To laze, waste time or take part in pointless labour
\item [Ey-Ohh] All purpose non-descript salutation
\item [Twatty] Adjective used to describe sections of cave which are tight, annoying or unnecessarily awkward.
\item [Schonky] Blanket adjective to describe anything dodgy, dangerous or loose, especially in caves
\item [Vitaminski] Powdered vitamin  sugary fizzy drink (comes in blue, yellow, orange and green flavours).

\begin{description}
	\item [slushies] Vitaminski mixed with ice from M10 on a hot day
\end{description}

\item [Blue Cloud] Patch of blue sky on a day of unrelenting clag
\item [Sunset] Evening ritual of fortified mountain tea consumption
\item [Slop] The evening cooked meal to go on the carbohydrates.
\item [the Hydro] A lovely swimming location near the Hydroelectric scheme of Tolminske Ravne
\item [$2^{nd}$ aid] Pills for hypochondriacs
\item [Old Lag] \textbf{O}wn kit, \textbf{L}eader, \textbf{D}river and experienced member of ICCC aged 25+
\item [Lightning Alley] A strip of fairly flat grass located high up on the Plateau. Not for the faint hearted.
\item [M10] Don't fall down M10
\item [Plateau] Undulating surface of Tolminski Migovec, anything between the Kuk and Mig trig points
\item [Mig] our spiritual home
\item [Union/La\v{s}ko polemic] Argument about the various merits of the two available beers in Slovenia. \sidenote{They are from the same manufacturer}

\end{description}

