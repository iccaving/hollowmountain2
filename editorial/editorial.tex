\chapter*{Editorial}

The cave exploration of \passage{Tolminski Migovec} from 2007 to 2012 has in
retrospect an obvious narrative. The story is of rags to riches, from a
seemingly hollowed out mountain long past its glory days of exploration
to the longest cave in Slovenia. The experience of living through these
times, is a more confused and complicated picture. We were individual
actors, with no script script, pulling together in some vague common
direction, all with the same overall objective, but some very different
ideas about how to get there.

If cave exploration were a simple, rational, expenditure of efforts
towards a known end, the scenes would be simple to describe. We realised
how close \passage{Vrtnarija} and \passage{Kavkna Jama} (M2) were, after more
carefully analysing the 2007 \passage{Kill 'em All} survey data. To pursue
the obvious potential connection, we rebolted and rerigged \passage{Kavkna Jama}
in 2008, while also exploring on the other side in \passage{Vrtnarija}. In 2009 we established a camp in \passage{Vrtnarija} at the nearest suitable point to the closest approach, and used it to massively increase our time at the pushing front. In 2010, 2011 and 2012 we camped deeper in \passage{Vrtnarija},
while pushing \passage{Kavkna Jama} from the surface both during the Summer and on
Autumn / Winter trips. In 2012 we connected the systems, forming the
longest cave system in Slovenia, and one in which the vast majority of
cave passage is at depths greater than 500 m.

However, that isn't the real story of the exploration. The story of the
people involved is the true history of \passage{Sistem Migovec}. The connection was not
made in the obvious location between \passage{Kavkna Jama} and \passage{Vrtnarija}, but down
at 650 m of depth, as the result of yet another successful, to the point
of routine, pushing trip. So what were we doing there?

Motivated by the connection target, during 2008 we flung ourselves back
into the exploration of \passage{Captain Kangaroo}, \passage{Vrtnarija}. At the grim pushing front were the youngsters, highly motivated but lacking the experience to go deep. Lacking in time at the pushing front we determined to go back and camp in 2009. 

``The art of roughing it is in smoothing off the edges.'' Stories of draughty campsites, cassette players slurring to an undignified
quietness, shivering through the night, and unlabelled plastic bags of
miscellaneous white powder were retold by the experienced members, and
duly obsoleted by careful consideration of the logistics. We went back
with free standing tents, layers of fleece, MP3 players, modern
winter-mix gas stoves and LED fairy lights. We went back to stay. Almost
effortlessly, we pushed this tough branch of the cave down to 550 m.

This new generation of cavers, who cut their teeth in \passage{Captain Kangaroo},
suddenly found themselves with the endurance and know-how to
successfully explore at depth. Though the connection of the systems were
certainly still a major aim of 2010, 2011 and 2012 expeditions, we were
mainly there to push deep new cave passage. We re-established \passage{Camp X-Ray} (550 m deep) as our main base in 2010. We improved on it year after
year, making it truly palatial. And now that the going was once again
deep, we were rejoined by the more experienced members of our club, for
whom the prospect of another grim rift in Captain Kangaroo had not been
suitably motivating.

As our collective abilities improved, normality shifted. Exploring over
multi-day camping trips, hot bunking and the considerable feat of
endurance just to reach and return from these depths became standard
practice. That which was just-possible the year before became the
standard trip, that which was beyond our reach became achievable.

I am proud of the time that I have dedicated towards these expeditions,
and every moment spent with the people involved. There are others in the
club who have contributed very much more. We were all volunteers. We did
all this because we wanted to, but little gets done if you only do
things that are fun.

Spending your free-time down caving stores fettling kit is neither
particularly enjoyable nor directly rewarding. Carries in the hail and
rain are arduous and unpleasant. I don't think it is possible for this
document to understate the sacrifice of time and effort made by
expedition members and friends. Forever lacking in adequate funding and
gear, unrecognised and often misunderstood, we explored with the bare
minimum.

This exploration report is dedicated to our many friends who assisted,
sponsored, carried, hosted and advised. You all contributed to the
achievements documented herein.

And so, Ninety-Nine years after Apsley Cherry-Garrard returned to South
Kensington with a Penguin's Egg, we returned to our college in 2012 with
a minor news story and a few pretty photos for their website. For those
involved in the exploration of \passage{Sistem Migovec} far more precious are
the memories of friendships formed deep within the \passage{Hollow Mountain}. The
prize was not the destination we arrived at, but the path we forged in
getting here.

We were always in the longest cave in Slovenia, we just hadn't realised.

\name{Jarvist Moore Frost}

\newpage
\begin{verse}
We shall not cease from exploration  \\
And the end of all our exploring  \\
Will be to arrive where we started  \\
And know the place for the first time. 
 \\
Through the unknown, unremembered gate  \\
When the last of earth left to discover  \\
Is that which was the beginning;  \\
At the source of the longest river  \\
The voice of the hidden waterfall \\
\end{verse}

\name{T. S. Eliot}