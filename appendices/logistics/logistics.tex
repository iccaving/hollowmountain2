\chapter{Appendix A --- Expedition
Logistics}

One aspect that is often missing from expedition writeups is the
expedition logistics that enabled the exploration. For sure, this
information will be the fastest in this publication to age. For though
the cave endures, and the human experience is understandable from year
to year, technical progress makes a mockery of our carefully considered
preparations.

During the period 2007--2012 covered here, we saw the total domination
of LED lights, displacing the previous carbide flames. Initially these
were mainly powered by Alkaline Flatpacks (helmet mounted) which cave
way to NiMh rechargeable packs as the brightness (and power input)
increased, and the length of our trips extended.

Our cave was rigged for Alpine style SRT, on Nylon rope. The main pitch
routes were rigged on 11 mm. Pushing was done on a mixture of 10 mm and
9 mm. Rope brands were selected for their toughness and ability to
resist abrasion more than ease of knot tying. Most of the caves were
rigged during this time on Mammut.

The majority of bolting was with hand-bolted Spitz (mostly to Petzl
twist hangers), giving way in recent years to 8 mm stainless through
bolts married with Raumer Stainless twist hangers. 7 mm long steel
maillons are used almost invariably for rigging.

There was approximately a 50:50 split with people wearing plastic
oversuits versus those of cordura. Most people caved with a normal
fleece furry and thermals, some stripping off for the prussic out.

\name{Jarvist Moore Frost}


\section{Three Years at X-Ray: Underground Camp
Logistics}

The Vodna Sled 2010 4-berth underground camp was extremely comfortable
and provided an excellent base for extended deep cave exploration.

As there seems to be little information written about setting up alpine
caving camps, we describe in this document an overview of the equipments
used, and resulting performance. This camp was used almost unaltered in
2011 and 2012, and we have included a few updates and tweaks gained
during those expeditions.


\section{Cave Conditions}

Vrtnarija is a typical deep alpine cave system.

The temperature measured at camp varies between 2 and 5 degrees
centigrade.

Camp X-Ray has a fairly considerably draft which flows from the (wet)
Zimmer pitch.

We would estimate the relatively humidity to be above 80\%, and note
that non-sealed paper becomes damp overnight.

\#\#Sleeping Arrangements

\#\#\#Tent

An extremely cheap 4-person single-layer dome tent was purchased from
eBay.

The tent fabric was washed at 60C in a large washing machine with an
excess of detergent in order to remove the water repellent coating and
thus reduce condensation.

This appeared to have been entirely successful -- no beading was
apparent.

The tent notably increased temperature and comfort at camp.

It was found impossible to close the doors fully due to the feet of
anyone above about 1.6m poking out the foot of the tent, but having the
bottom zip open was found to produce suitable airflow.


\subsection{Sleeping Bags}

Two of our berths were 1990s Buffalo bag fibre-pile liners, supplemented
with 200g sqm polartec fleece liners (supplied a long long time ago as
sponsorship in kind).

Most campers also required the wearing of a full set of fleece thermals
within these bags to remain suitably warm (Beast Sponsorship in 2009).

It was also difficult to actually get within the multiple layers of
sleeping bag, and one found oneself rather constrained once there.

By comparison, two of the beds were made out of Nitestar 450 synthetic
bags, purchased for circa. £30 each. T hese were found to be warm enough
on their own, though small girls in particular had a more comfortable
night when wearing fleece pyjamas.

A suggestion for future underground camps is to add synthetic silk
(nylon) liners to further increase the warmth.

The bags weigh 2kg each, but are extremely bulky.

Packing the bags back in London, we were able to fit the sleeping bag
and fleece pyjamas in one large oval tackle sac.

For the derig, we only managed to pack the sleeping bag alone into the
same large tackle sacks.

Nb: We replaced the Buffalo bags with Nitestars Sleeping bags in 2011.

The later 2011 edition Nitestar 450s are entirely synthetic (no cotton
in the liners) and thus almost the perfect underground camp sleeping
bag.

They feel noticeably less damp and sticky on the skin when you first get
in them.


\subsection{Roll Mats}

We now use `Nato 5 season' roll mats produced by Highlander / Outdoors
value for circa. ??10. They are long enough for the 2m tall folk.

\textbf{Colour: Olive green} Size: Open: 180 x 50 x 1cm, Rolled size: 50
x 15cmWeight: 300g

Superb compression recovery, Density: 25kg/CBM


\subsection{Condensation}

Condensation was minimum except for underneath the rollmats, as is
common for camping in cold conditions, and a slight temporary damping of
the top of the rollmat underneath the sleeping bag head.

One thing that was avoided was the careless use of superfluous fleece
camp clothes as a pillow -- it was found that this material provided a
wick for condensation.


\section{Cooking Arrangements}

Cooking at underground camp consisted of a Mini Trangia; recycled MSR
aluminium windshield for the Trangia; Campingaz Micro Plus Gas Stove;
`SunnCamp Trekker 5 Piece' Aluminium nesting cook pots (17cm and 18cm
sizes, including the 19cm lid / frying pan); clasping pot handle; 4
`lightmyfire' nylon sporks.

All this was packed into the largest 18cm saucepan and weighed circa
\textasciitilde{}1.5kg.

In general the trangia burner was used with the largest saucepan to cook
the breakfast / supper meals and was found to be sufficient for 4
people.

The medium saucepan was kept clean (ish) to be used to make hot drinks.

The small trangia saucepan was used to make small drinks (for instance
herbal tea / coffee when others were drinking black tea), and for
particularly dietry requirements (vegan) or simply to hold cut up cheese
/ salami during preparation.

2011: The `Campingaz' stove was replaced with a cheap `universal screw
fitting' then used with Primus Powergaz 4-season gas mix (we noticed we
were ending up with a frozen slurry of unburnt gas in our normal summer
mixes).

This new gas setup is really quite powerful, perfect for a quick hot
drinks.

Usually drinks are made as soon as cavers return / wake, drunk while
sitll undressing, with the food more slowly cooked on the Tranja.


\section{Food \& Drink}

Fish, Cheese, Soup and Smash were the general, standard permutations.

However, there was also significant quantities of instant noodles
(Sainsbury / ASDA own brand), CousCous (in particular the Ainsley
Harriet branded flavoured variety) and even Risotto mixes.

Other cooking ingredients included dried mushrooms and dried tomatoes ,
vegetable bouillon mix, miso soup mix and sesame seeds.

Condiments included smoked paprika and black pepper which had been
freshly ground on the surface and transported underground in a 35mm film
canister.

`White Powders' and other such bulk ingredients were taken down in
ultra-strong resealable plastic bags (100micron -- bought from
`thermalpaper' a dedicated plastic bag ebay.co.uk reseller), with the
contents written on in clear black marker pen.

Drinks, almost always warm or hot, were based on black tea (Yorkshire
Tea), local herbal teas (in particular Sadni Chi), hot chocolate (Makro
own brand) and Vitaminski (an effervescent flavoured vitamin drink
actually called `Cedevita').

Lunches were generally the standard caving snack food (chocolate bars,
midget gems, peanuts -- in particular honey roasted from Lidl), but also
supplemented with oatcakes and bread with salami, cheese and fish.

Spirits were taken down in 500ml plastic bottles and used as a small
nightcap by the majority of cavers.

The rolling hot-bed camp meant that every 12 hours all underground
cavers were physically present at camp, and therefore had their callouts
reset on a rolling basis.


\subsection{Saving Fuel \& other camp
craft}

A considerable number of tricks and tips were taught by the seasoned
expeditioneers to save on fuel and increase enjoyment at underground
camp. All simple, but useful, ideas.

\begin{itemize}
\item
  Smash doesn't need boiling water to make.
\item
  Noodles require boiling water, but can be cooked in a small volume of
  water, then have cold water added along with Smash to thicken.
\item
  Tea can be more efficiently made by boiling half the required volume,
  making strong tea, then mixing 50:50 with cold water to make an
  immediately consumable drink.
\end{itemize}


\section{Music \& Entertainment}

Music was provided by a Sansa Clip+ MP3 player wired into a pair of
folding travel speakers.

The travel speakers could operate of 4 internal AAA batteries, but were
found to be more powerful and longer lasting in the cold cave atmosphere
when powered over USB wired directly into a battery pack of 4 AA Eneloop
NiMh cells.

\begin{itemize}
\item
  Similarly, the MP3 player was recharged from a 2xAA NiMh
  ---\textgreater{} USB `emergency phone charger', but was found to be
  happy to charge off the unregulated eneloop battery pack as well.*
  2011: We moved entirely to just using the 4AA Eneloops + PP3 clip /
  micro USB adapter to directly power both the speakers \& recharge the
  Sansas.
\end{itemize}

As well as music (of various tastes!) audio comedy has been a mainstay
of underground camp, particularly in the evenings before falling asleep.
Blackadder, Father Ted, Dead Ringers, Little Britain, League of
Gentleman, The Mighty Boosh and the Ascent of Rum Doodle have all proved
popular over the years.


\section{Ambience}

Cheap tea-lights were taken down to camp and festooned on the cracked
rock walls around the tent.

A couple of stubby `church' candles were also brought down (bought from
`Tiger'), and were found to endure the cold atmosphere better than the
tea lights (which tend to burn a hole through the core rather than burn
all the wax).

This was reassuring, particularly for first time campers, and offered
reassurance and sufficient light to go for a pee.

2011: We added `AA battey' powered white fairy lights. Thes e were
bought cheaply from dx.com, and were found to run (via resistor
limiting) for an almost infinite time at a very low level, just enough
to orientate oneself when waking at night. The most pleasant ones were
`warm white' which had a very candle-flame like glow, even with mostly
depleted batteries.

\#\#Toiletry

Excrement was deposited directly into compostable corn-starch bags, of
the size used as standard compost bin caddy's and bought from a local
Sainsburys.

They were generally considered as `single use' -- except for when
supplies ran rather low towards the end! These were then tied together,
sealed in an additional non-biodegradable freezer bag and kept in a
Daren drum.

Standard rolls of toilet paper were taken down, but kept in a resealable
plastic bag to prevent damping in the cave atmosphere.

A alcohol based gel hand sanitiser was used for obvious reasons of
hygiene.

Once suitably full, the Daren drum was portered out of the cave, and the
biodegradable contents emptied into the latrine on the mountaintop.

*Not entirely sure where this came from (clearly been written by me).

Format appears to be Markdown, which implies it was destined for the
Caving website??? I???ve added a few notes from 2011 and 2012 (it was
written after 2010 evidently) and corrected a few typos.

Publish \& be damned, rather than have it sitting on my hard disk
longer. \textasciitilde{}Jarv*

name: Jarvist Moore Frost
