\section{The Eventful Conquest of Plopzilla
(nee.}\label{the-eventful-conquest-of-plopzilla-nee.}

Plop)

After my first trip to \emph{NCB} I was kept awake thinking about that
three to four second drop known simply as `Plop'. By eleven the next
morning I'd managed to convince Andy of the merit of a return visit.
Since we'd left the necessary tools and rope for bolting a monster pitch
in \emph{NCB}, we quickly shot down the \emph{M16} entrance series and
up Faulty Towers into \emph{NCB}.

Fairly terrified of getting stuck in the tight pitch head above that
formidable drop, I took off most of my SRT kit, leaving just cows-tails
and Croll. The squeeze was fairly easy and Andy passed through the bags
as I put in a bolt to make a Y-hang.

Since Plop had already been attempted several times there were quite a
few existing bolts. I made use of these on the way, stopping only to
take down a couple of boost bars: a bit of Cadbury Courage. I felt oddly
calm swinging about in the huge chamber. We had thrown more rocks from
the top but the bottom was too far away to see. Even from the first
rebelay I was having trouble speaking to Andy. Our words boomed around
the huge pitch. Two rebelays down, I was standing on a gravel floor,
shivering with the adrenaline. I'd been forced to put in a knot pass in
the rope and reverse prussic past it. Two more bolts got us to the
bottom of the pitch, by which point our nerves were totally shredded.
Though the hang of the rope was very clean, a rope disappearing into
empty blackness above can be really terrifying!

Though we were almost expecting to break into `Level 3', an as-yet
undiscovered horizontal passage at least three kilometres long, the
pitch was completed choked with boulders at its lowest point. We scoured
the nooks and crannies before pronouncing the bottom of Plop officially
dead.

However, twenty metres back up the gravel slope, another boulder choke
went down, an obvious lead for a return visit but by this stage we were
too tired to push and survey a new cave passage. We left a going lead in
boulders, along with an easy swing into a window halfway down the pitch.

Exit from the cave was difficult due to being tired and thirsty but we
were in a jubilant mood after a seventy-six metre survey leg! Plop was
the biggest pitch either of us had ever seen. As the first to bottom
this monster, we renamed it `Plopzilla'.

Analysing the survey data back at the bivi, it measured in at an
impressive 105 m of depth.

\name{Richard Venn}
