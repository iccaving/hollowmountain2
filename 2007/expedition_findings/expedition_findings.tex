

\section{Exploring the longest cave in Slovenia}

We'd made a name for ourselves last year, connecting the `old' \passage{System Migovec} with the newer \passage{Vrtnarija} (\passage{Gardeners' World}) system and the pressure was on to perform again. We certainly couldn't allow \passage{Postojnska Jama} any chance of reusing their pre-2012 literature (now the `longest cave in the \passage[Kras]{Classical Karst}'). 



\section{Expedition Findings}

\subsection{Plop Goes!}

Andy and Rik pushed \emph{Plop} (the tight squeeze) onto the magnificent
Plopzilla pitch. A field of helictites festoon the pitch down onto an
enormous boulder pile. One side of this chamber is unpushed, the other
leads to a boulder choke, as yet unpushed. \emph{Plopzilla} is 105 m
deep, penetrating from \emph{NCB} to below \emph{Exhibition road}. This
makes it the second largest pitch in the system after \emph{Silos}.

\subsection{M1 \& M6}

Repushed + resurveyed. Still a lead (may need chemical persuasion) in a
window off M1. Small extensions found in M6 - very pretty little bit of
stream formed new cave, ended in draughting bedding plane dig.

New caves on western edge of plateau

\emph{Planika} (named after the Edelweiss present on the wester
plateau)and ``Monatip'', found below B9/\emph{M2}1 (on the western edge
of the plateau, approx 100 m north of \textbf{Primadona}) and the
initial pushing trips conducted.

\emph{Planika:} 166 m long, 46 m deep \emph{Monatip:} 196 m long, 28 m
deep

\emph{Planika} is a high entrance (1801 m) which leads directly to a 40
m pitch to a snow plug. Climbing the snow gains another chamber with a
large entrance, and a rift leading off. A very tight pitch head at the
end of rift leads to a 5 m pitch which reconnects to a 20 m long snow
slope. Digging at the bottom of this snow slope gained another snow
filled chamber, with `phreatic' esque passage etched through the snow by
the draught, and extremely drippy snow which I believe is certainly
feeding a faithful stream, possibly the one that was found on Smer0 in
\emph{Primadona}, 200 m below. To get to Planika, one must conduct a 30
m abseil down a cliff, then traverse along the ledge. Extremely pretty -
one gets a view across the whole of the Tolminka by day, or the lights
of Italy all the way to Venice by night.

\emph{Monatip} is directly below Planika (24 m between bottom of snow
plug and early passage), and has a very \emph{NCB}-like character -
possibly a dried river bed. It undulates along, heading into blank
mountain

\subsection{\texorpdfstring{U-Bend connected to
\emph{Primadona}}{U-Bend connected to Primadona}}

Hard pushing by Sandeep and Alvin through the previously blown Enigma
squeeze on the 40 m U-bend pitch has led to a connection with the Druigi
entrance to \emph{Primadona}, gaining \emph{Primadona} an additional 57
m of height, making a cave 644 m deep. Beautiful survey accuracy - have
a look on the .3d file!

\subsection{\texorpdfstring{\emph{Razor} cave
survey}{Razor cave survey}}

Coordinated by Martin, we've started to survey \emph{Razor} cave. 250 m
already in the book, its a very interesting clearly fault-driven cave,
with easy access from the \emph{Razor} hut.

\subsection{\texorpdfstring{\emph{Primadona} Smer0
pitch}{Primadona Smer0 pitch}}

Initial rigging of the Smer0 pitch discovered in October 2006 was
undertaken, bolting down to \textasciitilde{}-40 m. Pitch is ongoing.
Stream was followed upstream to a tight labyrinth.

\subsection{Smashed Swede}\label{smashed-swede}

Stefan's climb was bolt-traversed to by Rik + Paul, gaining a window
that would appear to reconnect to \emph{Hardy} Pitch. A second look
wouldn't go amiss.

\section{Minor Caves}\label{minor-caves}

\subsection{East Pole (S1)}\label{east-pole-s1}

Further work was undertaken in East Pole: a number of promising new
holes were investigated, including E1 - at \textasciitilde{}25 m deep
pitch leading to too-tight windows that require opening up.

\subsection{Stag Cave}\label{stag-cave}

A cave was found within 20 m of the tents! A short pitch, rigged on
naturals, lead to a spacious chamber that was unfortunately dead.
However the presence of a large collection of bones (some crushed, but
many in very good condition) that appears to have been from a stag made
up for the disappointment!

\subsection{Moth Cave}\label{moth-cave}

Heroic effort was expended in the Moth Cave dig: two extra chambers were
gained, but unfortunately only lead to yet another too-tight squeeze
requiring rock removal. Declared dead and derigged.

\subsection{Hawk Cave}\label{hawk-cave}

New (safe) method of gaining the cave was constructed by bolting an
abseil from the cliff-head. Most leads off chamber were found to die, a
bolt traverse was made across the pitch to find an aven where we hoped a
parallel shaft may lie. Still to be revisited :- we ran out of time and
rope, and so derigged.

Most of all, this expedition was an enormous training mission: we now
have an extremely strong expedition team together once more, with great
ties to the new JSPDT members.

I think that all the lags can feel extremely proud of the enormous
cannon of information that has been passed on, the new members proud of
the steep learning curve that they all conquered, and everyone proud of
the Caves, little and big, deep and shallow that we've found this year.

\name{Jarv}
