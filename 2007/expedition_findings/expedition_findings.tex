\section{Expedition Findings}

\margininbox{Logbook}{W566 BFR is our bus of joy, 5.6 metres long

\textbf{Ferries:}

Out on Sat 14th July, 1345 ish, back Sun 12th August, 1100 ish.

Quotes (8-12 weeks in advance): P\&O (9 passengers): 156.25 

With Sea France: 112.00

Sea France (4 weeks in advance): 140 \textbf{BOUGHT}
\protect\mininame{Jarv}}{\logbook}

\begin{marginfigure}
\frame{\includegraphics[width=\linewidth]{2007/expedition_findings/jarvist frost - sandeep and cheesus in stores--orig.jpg}}
\caption{Sandeep and cheesus [sic] in stores. \pic{Jarvist Frost}}
\end{marginfigure}

\subsection{\protect\passage{Plop} Goes!}

Andy and Rik pushed \passage{Plop} (the tight squeeze) onto the magnificent
\passage{Plopzilla} pitch. A field of helictites festoon the pitch down onto an
enormous boulder pile. One side of this chamber is unpushed, the other
leads to a boulder choke, as yet unpushed. \passage{Plopzilla} is 105 m
deep, penetrating from \passage{NCB} to below \passage{Exhibition road}. This
makes it the second largest pitch in the system after \passage{Silos}.

\subsection{\protect\passage{M1} \& \protect\passage{M6}}

Repushed + resurveyed. Still a lead (may need chemical persuasion) in a
window off \passage{M1}. Small extensions found in \passage{M6} - very pretty little bit of
stream formed new cave, ended in draughting bedding plane dig.

\subsection{\protect\passage{U-Bend} connected to \protect\passage{Primadona}}

Hard pushing by Sandeep and Alvin through the previously blown \passage{Enigma
squeeze} on the 40 m \passage{U-bend} pitch has led to a connection with the \passage{Drugi Vhod}
entrance to \passage{Primadona}, gaining \passage{Primadona} an additional 57
m of height, making a cave 644 m deep. Beautiful survey accuracy - have
a look on the .3d file!

\subsection{\protect\passage{Razor Cave} survey}

Coordinated by Martin, we've started to survey \passage{Razor Cave}. 250 m
already in the book; it's a very interesting clearly fault-driven cave,
with easy access from the Razor hut.

\subsection{New caves on western edge of plateau}

\begin{pagefigure}
      \checkoddpage \ifoddpage \forcerectofloat \else \forceversofloat \fi
      \centering
              \frame{\includegraphics[width=\linewidth]{2007/expedition_findings/jarvist frost gr1 film1 -029_26--orig.jpg}} 
  \caption{The impressive \passage{M1} entrance scree slope. \pic{Jarvist Frost}}
\end{pagefigure}

\passage{Planika} (named after the Edelweiss present on the western plateau) and ``\passage{Monatip}'', found below \passage{B9}/\passage{M21} (on the western edge
of the plateau, approx 100 m north of \passage{Primadona}) and the initial pushing trips conducted. \passage{Planika}: 166 m long, 46 m deep \passage{Monatip}: 196 m long, 28 m deep.

\begin{marginfigure}
\frame{\includegraphics[width=\linewidth]{2007/expedition_findings/jarvist frost gr1 film4 -032_29--orig.jpg}}
\caption{The snow-filled first chamber of \passage{Planika}. \pic{Jarvist Frost}}
\end{marginfigure}

\passage{Planika} is a high entrance (1801 m) which leads directly to a 40 m pitch to a snow plug. Climbing the snow gains another chamber with a large entrance, and a rift leading off. A very tight pitch head at the end of rift leads to a 5 m pitch which reconnects to a 20 m long snow slope. 

Digging at the bottom of this snow slope gained another snow filled chamber, with `phreatic' esque passage etched through the snow by the draught, and extremely drippy snow which I believe is certainly feeding a faithful stream, possibly the one that was found on \passage{Smer0} in \passage{Primadona}, 200 m below. 

To get to \passage{Planika}, one must conduct a 30
m abseil down a cliff, then traverse along the ledge. Extremely pretty - one gets a view across the whole of the Tolminka by day, or the lights of Italy all the way to Venice by night.

\passage{Monatip} is directly below \passage{Planika} (24 m between bottom of snow
plug and early passage), and has a very \passage{NCB}-like character -
possibly a dried river bed. It undulates along, heading into blank
mountain.

\begin{marginfigure}
\frame{\includegraphics[width=\linewidth]{2007/expedition_findings/jarvist frost - planika first snow chamber - 8m deep hole in snow--orig.jpg}}
\caption{An 8m deep snow hole in \passage{Planika}. \pic{Jarvist Frost}}
\end{marginfigure}

\begin{marginfigure}
\frame{\includegraphics[width=\linewidth]{2007/expedition_findings/jana carga - jarv near rebelay entrance pitch planika--orig.jpg}}
\caption{Jarv descending near the first rebelay in \passage{Planika}. \pic{Jana Čarga}}
\end{marginfigure}


\subsection{\protect\passage{Primadona} \protect\passage{Smer0}
pitch}

Initial rigging of the \passage{Smer0} pitch discovered in October 2006 was
undertaken, bolting down to about -40 m. Pitch is ongoing.
Stream was followed upstream to a tight labyrinth.

\subsection{\protect\passage{Smashed Swede}}

Stefan's climb was bolt-traversed to by Rik and Paul, gaining a window
that would appear to reconnect to \passage{Hardy} Pitch. A second look
wouldn't go amiss.

\section{Minor Caves}

\subsection{\protect\passage{East Pole} (\protect\passage{S1})}

Further work was undertaken in \passage{East Pole} {\passage{S1}). A number of promising new
holes were investigated, including \passage{E1} - a roughly 25 m deep
pitch leading to too-tight windows that require opening up.

\subsection{\protect\passage{Stag Cave}}

A cave was found within 20 m of the tents! A short pitch, rigged on
naturals, lead to a spacious chamber that was unfortunately dead.
However the presence of a large collection of bones (some crushed, but
many in very good condition) that appears to have been from a stag made
up for the disappointment!

\subsection{\protect\passage{Moth Cave}}

\begin{marginfigure}
\frame{\includegraphics[width=\linewidth]{2007/expedition_findings/jarvist frost - jana entrance to e1--orig.jpg}}
\caption{Jana in the entrance to \passage{E1}. \pic{Jarvist Frost}}
\end{marginfigure}

Heroic effort was expended in the \passage{Moth Cave} dig: two extra chambers were
gained, but unfortunately only lead to yet another too-tight squeeze
requiring rock removal. Declared dead and derigged.

\subsection{\protect\passage{Hawk Cave}}

New (safe) method of gaining the cave was constructed by bolting an
abseil from the cliff-head. Most leads off chamber were found to die, a
bolt traverse was made across the pitch to find an aven where we hoped a
parallel shaft may lie. Still to be revisited - we ran out of time and
rope, and so derigged.

Most of all, this expedition was an enormous training mission: we now
have an extremely strong expedition team together once more, with great
ties to the new JSPDT members.

I think that all the lags can feel extremely proud of the enormous
cannon of information that has been passed on, the new members proud of
the steep learning curve that they all conquered, and everyone proud of
the Caves, little and big, deep and shallow that we've found this year.


\name{Jarvist Frost}

\begin{pagefigure}
      \checkoddpage \ifoddpage \forcerectofloat \else \forceversofloat \fi
      \centering
              \frame{\includegraphics[width=\linewidth]{2007/expedition_findings/jarvist frost gr1 film3 -029_26--orig.jpg}} 
       \label{surface bashing}
  \caption{A surface, quite literally being bashed by James Huggett. \pic{Jarvist Frost}}
\end{pagefigure}

