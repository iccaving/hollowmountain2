
\section{\texorpdfstring{Rerigging
\emph{Vrtnarija}}{Rerigging Vrtnarija}}

An early morning on expedition\ldots{} I stumble my way down to the
Bivi, late as ever, and set about fixing some coffee. It's a disturbing
hive of activity. William and James KP are planning to rig as far as
possible, replacing the rope below \emph{Pico} with new and rebolting
where necessary, Tetley is following them down and introducing the new
cavers to \emph{Vrtnarija}.

Excellent! Everything is in hand. Gergely suggests we follow up the rear
with extra rope and rigging gear, preparing the ground for tomorrow's
deep pushing. Looking forward to a nice, relaxing, afternoon trip, I
don't bother to cram food into myself (I'm never that innately hungry in
the mornings), and Gergely and I take the minimum of provisions.

It's nice to be back in \emph{Vrtnarija} again, after a year, and even
the stepping aside on the pitches to allow the vast quantities of cavers
to pass is fine. Eventually we pass everyone else \& get to Swing, and
find William and James KP. They're both pretty pissed off, putting a new
bolt in on Swing and rigging the blasted thing has sapped their energy.

So Gergely and I take over. I rerig Tesselator, coiling the old rope
ready for recovery. The head of space odyssey receives an extra bolt and
thus a Y-hang, originally I intend to remove the deviation entirely, but
find that I've misjudged slightly and it's still required, just.

The traverse ledge halfway down space odyssey needs more work. Gergely
comes down and we work on it together, me putting in more bolts and
working new rope out along the traverse, whereas Gergely carefully
walked out on the old traverse and rigged the pitch down to Concorde. We
have a bit of confusion in the middle, as we also have to start a new
rope. All sorted out, we have a spare 15 m of traverse rope that we hide
for future use in a cubby hole.

Leaving the bolting kit and rope for future riggers, we have a smooth
exit. Rather deeper and more effort than I had been intending for this
first day's trip, and I was certainly feeling the lack of food as I
prussic'ed back up the entrance series, but successfully completed
nonetheless. In their first day of caving the expedition had rerigged
down to -300 m!

\name{Jarvist Moore Frost}


\subsection{\texorpdfstring{The discovery of
\emph{Wonderland}}{The discovery of Wonderland}}

It was another rainy day. Arriving not too long ago to the plateau, the
thrill of the new possibilities for discovery was boiling in my venes.
While drinking incountably many cups of tea, we discussed the
possibilities with the old lags who are always the source of infinite
wisdom. Dave suggested to check out \emph{Leopard}, the passage that
opened from \emph{Zimmer}, just opposite of \emph{Friendship Gallery}.
Martin McGowan had been there some 10 years ago, but information was
unclear, and the chances for the continuation there had been at least
dubious -- so given it is so close to camp, why not give it a go?

Andy was sort of up to going underground, and I felt super happy to go
caving with one of the heroes of the discovery of many underground
passages on Mig. So the usual endless faffing started, packing good
amounts of food, Vitaminski, gas canisters, and some Zganje, and
finally, well in the afternoon, we started our descent towards the
unknown.

The way down we got again used to the feeling of hanging in big abysses
on dubious bolts, which proved to be quite useful for what followed.
Reaching \emph{Zimmer}, Andy chatted to Tetley and Myles (who were
already at camp), while I bounced with the rope to the window and put in
two bolts for having an easier start the next day. Having a sleep at
camp was OK, although Andy did not seem to be very happy about the
limited comfort of \emph{X-Ray}, and he stated that he was too old for
underground camping -- a fact that changed little of our plans for the
next day, of course.

So, in the morning we climbed up to the gallery, and secured the rope to
a large boulder at the top of the climb. \emph{Leopard} was very nice, a
typical phreatic tube, with formations resembling the spots of a
\emph{leopard} on its walls, and also with a fairly good draught (I am
always keen on following the wind). At the other end, a dark muddy
pitch-head waited us, with a good number of scary, loose boulders around
it. We again noticed that Gardeners' World is so aptly named, and
started to clean up the place. We could not remove the largest rock,
tilted against the wall, so decided to go underneath it. We were not
quite sure that it is going to stay, but it is still there, after
hundreds of cavers have passed beneath\ldots{}

I took the nice 9 mm rope that we packed, secured the end around a big
boulder in the passage, and started the descent. The slope was terrible,
completely muddy and also full of loose material. Luckily, I managed to
put in a bolt in a piece of bedrock on the right side that popped out
from the debris. Then, descending through the narrow window, a large
chamber opened up beneath my legs.

It was absolutely terrible. Everything I touched fell down instantly,
starting a small avalanche of loose rocks. I managed to put in one
deviation, and then, after many failed efforts to put in a rebelay, I
finally descended down to the bottom of the chamber, where, to my
amusement, I saw some little stalactites, and great passages starting
off. I cleared off from the bottom off the pit, and Andy followed very
carefully. When he reached the bottom, he asked me in his typical
sarcastic style, if I knew that a rub-point can cut a 9 mm rope already
during only one descent. Well, sure, but it already survived two, so no
problem! - I replied.

We built a cairn at the bottom for easier surveying, but no traces of it
remain now. The pitch caused quite a trouble for a long time -- due to
its unforeseeable nature, it later got the name ``\emph{Cheetah}''.
Finally, after some years' time of usage, it seems to stabilize, at
least let's hope so\ldots{}

So, we started to discover the big passages that laid ahead. Little did
we know about the many kilometres hidden behind them, but we felt that
we got into something different, something very old, and it was very
exciting. Andy recalled that he had found a lot of large vertical
pitches, despite the fact that he hates big pitches -- so finally, he
was eager to find something horizontal!

First, we went to the large passage that opened towards the right. We
followed it for about 200 meters, with small drops, and we finally
reached a bigger drop that lead to a chamber filled with rocks.
Unfortunately, our rope was too short (although we even made use of the
chords of our tackle-sacs), so we could not descend there. Mirroring our
surprise, we named this part \emph{Wonderland}, and went back to the
first pitch.

Starting in the other direction, I put in a small traverse line with its
middle belay at a small waterfall, and reaching the other end, we saw
that Andy's dream finally became true -- we found a true horizontal
passage, whose ceiling was covered with small crystals! (Since then,
experience showed that these are typical wind crystals, indicating the
way of strong draught -- and here, indicating the way towards the
Connection, although that is a different story\ldots{}) Hurrah! We
followed the passage eagerly, and we were even more amazed when it
opened up to a junction. A flat-out crawl lead to a nice little chamber,
whose ceiling was full of little crystals. On the left, we also found a
pretty water inlet, filled with white sand. Underneath the main passage,
we also climbed down to some openings where water could be heard in the
distance, perhaps 40 m below us.

Finally, we reached the place which seemed to be the termination of the
passage - a boulder choke. Hmm. We soon managed find a way upwards, and
when popping out at the top, we suddenly found ourselves in a big void!
Wow! A big chamber! With a waterfall! And a window! And another window!
And an obvious continuation of the phreatic!

Andy asked if I knew of \emph{Exhibition Road} in the system, and that
this was similar to that, so we decided to name it \emph{Prince Consort
Road} (later, quite aptly, it got renamed as \emph{Albert Hall}). By
that time, it was quite late, and we had to start back to survey, but we
did it happily. Finally, quite broken, we prussiked up on the very
dubious 9 mm rope, took it up so that no other person repeates the same
insanity, and walked back triumphantly to \emph{X-Ray}.

Andy's words in the logbook express the essence of the day: ``AJ and GA
turned one crappy lead (\emph{Leopard}) into lots of great leads''.
That's what perfect cave exploration should be, isn't it? Well, our luck
has proved to be good, once again\ldots{}

\name{Gergely Ambrus}


\section{\texorpdfstring{Pushing
\emph{Insomnia}}{Pushing Insomnia}}

I had travelled down to T'min to clean and get over the cabin fever that
develops over the weeks of life on the Plateau. That night, after a slap
up meal and a few bruskies, Jan arrived and we had a great evening of
beer and bullshit. After the beer, we passed on to the whiskey and the
bullshit got increasingly epic. ``The leads at the bottom of \emph{Red
Cow} are going to make GW deeper'' I told Jan ``the next pushing trip
will definitely do it''. So after walking up the hill having dinner and
a little too much to drink at the bivi we set off for the night train.

Soon enough we are at camp and decide to keep going down, looking to
continue pushing the \emph{Republika} lead. I must admit it took a lot
of strength not to push the leads that were already multiplying near
camp, but I knew the way to \emph{Red Cow}, having been there with Dan a
few days earlier. We soon got to the junction and followed the water
upstream. Nice caving, I start feeling the all familiar excitement: here
come lots of km of fresh cave!

At one small pitch I turn around and find Jan has disappeared. I turn
back to look for him and find him wandering in the wrong direction
towards the sump. Apparently he had fallen asleep and started wandering
off route. Shit maybe we are too tired for this? Meh! We get to the
pitch head for \emph{Republika}, which I must admit is rather God
forsaken and wet and awful. We drop the pitch, I get the drill out and
start rigging, brain totally disengaged. As I am rigging I can feel the
batteries getting weaker and weaker. I guess there were a few too few
bolts at the bottom of the main drop and maybe some of the pitch heads
could have been a little neater. It certainly helps to cave with a tall
bastard is all I can say :D.

We reach the bottom of the main new pitch and it's a rather God forsaken
wet and damp place. The water keeps going down along some immature
passage. We follow the water, noting at least one unlikely but unchecked
possible side passage. A few hand lines are placed here and there and
the drill battery finally dies. Just as well as the last pitch we get to
looks like a right nightmare: really tight pitch head etc. By this point
we realise it's super late and we are almost certainly going to miss our
callout. Time has gone in a blur, we are probably not 100\% there
mentally to be honest. Still might as well survey out.

The way out is not very remarkable. We bump into Tetley at the top of
\emph{Big Rock Candy Mountain}. He is not too worried, but apparently
Gergely was hoping we were crumpled up in a heap somewhere so he could
come and rescue us.

\name{James Kirkpatrick}


\section{More Horror with Jan.}

The next night we get up and decide to do some pushing. The whole day is
marred by the fact that - once again - we promised not to push the
obvious continuation from \emph{Albert Hall}. So we decide to have a
random bimble around. We try looking for alternative ways around the
\emph{Albert Hall}. No success. Somehow we end up pushing a lead from
the right of the passage leading to the \emph{Albert Hall} (right
looking towards the \emph{Albert Hall}!).

The passage is small and bends under the main route. It is wet and
rather grotty. We drop a pitch or two of utter horror and eventually
turn around. Surveying out is awful. The book gets wet, my fingers are
too cold to hold the
instruments\sidenote{The survey data had to be 'corrected' to avoid the survey self-intersecting due to backwards legs.}.
It sucks majorly. At least the passage gets a cool name:
\emph{Esoterica}. O, and no-one has pushed the pitch where we stopped!
\sidenote{\emph{Esoterica} was not revisited until 2012, where a broken bolting driving prevented any additional pitches being descended. 
It remains a lead.}

\name{James Kirkpatrick}


\subsection{\texorpdfstring{Happy days with Jan --- Visiting
\emph{Palace of King
Minos}.}{Happy days with Jan --- Visiting Palace of King Minos.}}

On our last day together we went looking around the amazing passage that
we politely left for Dan and Izi. We smashed our way through the
entrance of what would become \emph{Povodni Mo\v{z}}, we looked at the
Queen's bed chamber, we pushed down some random tubes at the sides of
the main route (has anyone ever surveyed these I wonder, one of them
went!). All in all a super chilled day. No surveying and no water. And
then we got back to camp, watched some videos and drank a lot of
whiskey, Happy Days!

\name{James Kirkpatrick}


\section{\texorpdfstring{\emph{Balamory}}{Balamory}}

When James and I went to have a look at the bottom of \emph{Balamory} on
the first day of our one-noght camping trip, we didn't make it to the
target as an unexpected first pitch used up too much of the rope that we
had taken with us.

However, as well as airflow going towards \emph{Balamory}, there is also
airflow in the main passage above, beyond the point where the hole down
to \emph{Balamory} leads off so here are two leads in that area there
that are worth looking at.

The upper lead \emph{does} need some way of climbing across a pit to a
ledge covered in loose crap (maybe a bolt traverse round one of the
walls, though I can't remember how much good rock there was in that
area.

Subsequent email exchanges on the subject do seem a bit vague as to
exactly what was done, and whether the upper passage had actually been
boldly examined or not (Clewin was bold somewhere round there, but it
wasn't certain where), but anyone going to potentially move/split the
boulder down \emph{Balamory} (I think it was supposed to be down the
third of the three possible `second pitches') down should also probably
plan to look at the upper passage continuation, especially if they have
a drill/bolts to help with climbing/traversing belays.

On the way back from this little trip, after grabbing some stashed
karrimats from the old camp, that the black black hole across \emph{Big
Rock Candy Mountain} was noticed.

The next day, James and I went to the new stuff below
\emph{Leopard}/\emph{Cheetah}, but since we were doing a push-then-out
trip, rather than going to the far end of the `half' where most of the
action was, we went in the other direction from the pitch bottom to do
some work relatively close to camp, carrying on where Gergely+???? had
left off at the \emph{Hidden Surprise} pitch.

Gergely had dropped the first section of the pitch to an obvious ledge
and followed the ledge to horizontal passage that soon died. James and I
were to descend the shaft from the ledge, but it immediately became
clear the drill wasn't working and so we had to resort to spits, with
James bolting while I waited on the ledge.

Fairly soon the bolt (bolts?) was in and James had descended to a large
boulder-covered ledge part-way down the large shaft, where I could
safely join him. Some of the boulders were rather large, with gaps
between them or between them and the wall large enough to climb down and
move around in. While James did the business with the next spit, I
wandered around between the boulders to keep busy. Where the boulders
met the wall, the wall was somewhat overhanging, and from the lowest
easily-accessible place near where I had first climbed down, it was
possible to look between the boulders marking the lower limit of easy
movement and the wall to see a few metres away/down to where the wall
seems to meet the proper floor of the ledge, where there was a layer of
white rock flour, with some potentially human-sized space between me and
it though with no obvious way to get there.

From where I was, looking along the wall `clockwise', it also looked
like there was a space of some sort ahead of me horizontally, but
getting to it didn't look very nice, and after all, I was just in a pile
of boulders on a big ledge half-way down a pitch, not in a classic
boulder choke as such, so there seemed little point doing anything
borderline just to get to a slightly different place in the boulder
pile.

However, just as I was preparing to go back up and see how James was
getting on, I breathed out a large sigh, only to see it get sucked
horizontally away from me between wall and boulders and into the space I
had been looking into, which immediately aroused my curiosity.

To get into the space I could see required going horizontally through a
not-quite-body-sized vertically-rectangular gap with a short (1 m) drop
on the other side. After removing all my SRT kit, and doing some work
with a convenient rock hammering edges off the boulder forming one side
of the slot to make the gap wider, and progressively blunting sharp
edges on the wall side as they proved awkward when attempting to get
through, it was possible to slowly and delicately post myself through
feet first and eventually emerge free on the other side.

Turning around, a short crawl led to a wider area under the overhanging
wall, and a view ahead to where the wall/roof sloped nicely down towards
into the floor to leave a wide bedding plane with clearly no way on.
Turning around somewhat disappointed, the main wall, which I had been
looking away from when I had initially turned round, was seen to have a
crawling-height hole in it, which, on approaching, it was clear most of
the draught was going into.

That hole led to a small chamber with a further hole leading in turn
into the side of a walking-height passage with a good breeze running
along it. The draught I had followed initially was clearly just a
tributary being sucked into the main airflow.

I quickly returned to James to tell him of the find, and we decided to
do a little surveying and exploration. We chose the upwind branch which
didn't run a great distance before ending in an upwards bedding-plane
slope ultimately blocked by a large slab in the bedding blocking
sideways movement into what appeared to be a chamber with a waterfall
entering. Capping or plugs/feathers would seem to be needed to shift
this blockage. On returning to our entry point, we looked the other way,
wondered how far the downwind passage went, but left it for someone else
to explore.

We hadn't found a great deal of length, but on the other hand, we had
left a decent going lead, and due to the combination of a misbehaving
drill making waiting cold and dull work and the luck of my breath
showing there was something worth looking at, had ended up finding quite
interesting passage in what must be one of the most unlikely of
situations.

Thinking partly of the initial nervousness with which I had slowly
posted myself between the boulders and the wall, but mainly of the
immense luck we had had with the draught, \emph{Kamikaze} seemed like
the obvious choice of name for the discovery.

\name{Dave Wilson}

Last year we manage to connect Captain K. with lower parts of
\emph{Vrtnarija} (\emph{Friendship Gallery}), so I could not wait to go
caving again, especially to first descent the whole \emph{Pico}.

Dan and I decided to go camping in \emph{X-Ray}, which was after
re-established after many years.

We were on the night train. So the plan was to go caving late at night,
reach the camp in the morning and then go straight to bed. When we
reached the camp we find a note, asking if we could wait a bit longer,
because the other team return late. We decided to go and check out the
newly discovered parts and maybe find some leads for tomorrow. When we
return to \emph{X-Ray} we meet our tent mates Gergely and Niko, who just
return from their caving trip. We prepared some food, change to dry
clothes and went into sleeping bags, while others started preparing for
caving. It was funny to watch how they were not happy to put on wet
caving clothes.

\name{Izi}


\section{Palace of king Minos, Queen Bed chamber, Minotaur rift and
Ouroboros}

We were woken up by Jarv and Jan who just return from their pushing
trip. While cursing wet clothes we discussed what they had found and
what was worth visiting and pushing. Soon we took off and separated on
the bottom of \emph{Cheetah}.

Gergley and Niko went down \emph{Wonderland}, Dan and I went to
\emph{Prince Consort Road}. Road lead us to \emph{Albert hall} where not
long ago Jarv and Jan were pushing Serpentine. We decided to push some
new leads, so we climbed higher in this chamber and from there we saw a
big gallery. To reach this gallery we needed to climb about 6 m high.
Dan gave me a push up and I was able to reach the top. There I secured
the rope around the bolder and Dan joined me. From there we made a
traverse and then entered the gallery. Already on the entrance we
noticed something amazing. The ceiling was covered with white crystals,
that we never accounted before. That meant this is an old phreatic
gallery and we were so lucky to be the first one to discover it. Also
the soil on the ground was brown on top and once we stepped on it, it
was white underneath. Our first steps left white trails of footsteps
behind, truly amazing. We were really careful with our steps, as we did
not want to make to much damage. We mainly stick to the main passage and
there were no pitches or tight squeezes, perfect for exploring. On the
end of this gallery we were stopped by an aven. We could not climb it,
so we decided to return and start surveying back. On the way back we
took another side passage which lead us back to \emph{Albert hall}.
Because we made a loop and the hole place look like a maze, we name it
Palace of King Minos, the side passage was named Ouroboros, cuz it was a
loop to \emph{Albert hall} and also we named the big rift Minotaur. We
left one side passage for next time, because we were already tired.
Surveying this gallery was a pure joy. Long distanced between stations,
minimum of 10 m. Both of us were extremely happy with our discovery and
so we return to \emph{X-Ray} for well deserved rest. In the camp we meet
the other team and also they find some new meters. We exchange our trip
experience and fall a sleep.

When we woke up, we were surprised by roaring in \emph{Zimmer}. During
the day it was raining outside, therefore a lot of water was purring
down \emph{Zimmer}. Dan and I still have one more night at a camp, but
because Niko and Gergely were not feeling to go out in these conditions,
we volunteered to do it. Beside, someone had to cancel their call out.
So we eat something and start going up to the surface. I was first
heading out, but before I fully attached myself on the rope in
\emph{Zimmer}, I was already all wet. I told Dan that I will not wait
for him and because we both know the way out, we will see each other on
the surface. I was going for about an hour, stopped, roll 2 cigarettes,
smoke one, and leave one for Dan and then continue out. Below
\emph{Pico}, I repeated the same procedure, so at that time Dan catches
me up and we continued together. Above \emph{Pico} it was less water so
that meant that the rain has stop. We reached the surface in 3 and half
hours. In the bivy we entered the data and we were so happy when we
heard that we surveyed more than 500 m, especially how easy was to do
it.

\name{Izi}


\section{\texorpdfstring{\emph{Hidden
Surprise}}{Hidden Surprise}}

Well, the name \emph{Wonderland} does not really exist anymore, but back
then this was used for the horizontal level at the bottom of
\emph{Leopard}. Once finishing our triumphant trip with Andy, I was
eager to get back to continue the exploration of the many passages that
lay below. So, we paired up with Niko, to form a truly Eastern European
team, and went back to do what we had to do -- rig the drop at the end
of the \emph{Wonderland} passage!

Izi and Dan went to the other end, to find the continuation from the big
chamber (\emph{Albert Hall}). With Niko, we improved some of the rigging
that we did with Andy, and then dropped to the chamber with loose
boulders. Unfortunately, the rope that I used was new, and it shrunk
considerably later, so those rebelays have been too tight ever since.
Within the chamber, we climbed down to the obvious continuation, to find
to our horror an incredibly loose pitch within a pile of incredibly
loose boulders -- so we agreed that this was not the best option, if we
still want to do something else with our lives. Nevertheless, this
experience provided us with a good name -- \emph{Rolling Stones}! So,
there we were, in a nice big chamber, with a suicidal way onwards, and
we were not quite keen enough to continue that way\ldots{} We had a look
here and there, to no avail, and were about to turn back, when Niko
climbed in between some rocks, and said in his usual slow manner --
``Heey maan, maybe this is the right waay, nooo?'' And indeed it was!
Following the draught, we soon managed to open up a small passage, which
lead to a flat-out crawl, but after some 20 m it again popped in to
another chamber, quite nicely decorated. \emph{Hidden Surprise}! That
was the name we chose. The way on was obvious, and soon we emerged at
the top of a large black void space. This was a good point to turn back,
so we surveyed everything and showed it proudly to the happily emerging
pair of Izi and Dan. Then we returned to camp together, shared a nice
meal, talked about our stories, and had a nice evening, as usual at Camp
\emph{X-Ray}.

Next day, we went back with Niko to the top of the pitch. Two ways were
at the offer: either down to the black, or across to the seemingly
obvious continuation of the phreatic tube. I was quite scared of going
down, so we chose the more safe option of doing the traverse. Niko put
in his first ever bolt at the Milka rock (which resembles a cow), and
then we slowly proceeded, given that the whole thing was done by hand
bolts. Finally I climbed up the slope on the other side (quite a scary
experience), and we were there at the start of kilometres of a wide,
walking phreatic passage! At least, we believed so. This expectation
proved to be immediately wrong, when after 15 m, we had to take off our
SRT kit in order to pass a squeeze. Never mind, we thought, the walking
passage is just on the other side! Well, not quite\ldots{} the ceiling
got lower and lower, and we really had to fight to advance about 50 m.
There, the passage became really low, so we decided to try to survey
back. It has been a gymnastic feast, not only small and low, but also
super uncomfortable because of the dry, fossilized mud formations that
filled the floor completely. Niko was super happy to be there -- the
afternoon was mostly passed by his constant ``Noo, maan''
moaning\ldots{} The mud formations are indeed very nice, maybe they
should be scientifically investigated. Anyway, they gave the place a
name -- Mudstone Squeeze and \emph{Mudstone Traverse} were what we
surveyed that day. To this date, nobody has been back to Mudstone
squeeze -- although it is quite a place to go, if you happen to have
some free time and want to exercise your squeezing and crawling skills!
:)

Back at the camp, we heard the great news: Izi and Dan just found 600 m
of passage at the Palace of King Minos! We were about to exit the cave
the next morning, but there apparently had been a great storm, making
\emph{Zimmer} impassable for a good 20 hours. So we got the opportunity
to have a tourist trip to King Minos Palace, and we did very well to use
it- and absolutely amazing bit of cave! After another night at camp, we
finally started to the surface after 4 days underground, but we were
certain to return to the plethora of exciting leads\ldots{}

\name{Gergely Ambrus}


\section{Discovering Povodni Mož}

After couple of days on the surface, I decided to do some camping with
my Slovenian mates. Tjaša, Erik, Mawer and I started to walk towards the
entrance of \emph{Vrtnarija} at around 2 am. We reached the camp too
early for our bed time
\sidenote{On the Night Train, the bed time is 8 am to 8 pm}, so we
decided to go and check out the crystals in \emph{Palace of King Minos}.
After this, we still had time so we decided to go and try digging in
\emph{Minotaur Rift}, where last time with Dan we could hear the
enticing `roar' of water. After couple of hours moving boulders with no
breakthrough, we returned to camp.

The next day we planned to investigate one lead in \emph{Palace of King
Minos}, that we had not yet pushed with Dan. The entrance to the lead
was quite narrow, but beyond it we reached a series of active water
pitches. We had enough rope to bolt two drops and decided to name the
passage \emph{Povodni mož} \sidenote{Water nymph}, as we had been
following the water all the time.

\name{Izi}


\section{\texorpdfstring{\emph{M2} --- \emph{Kavkna
Jama}}{M2 --- Kavkna Jama}}

The JSPDT organised a trip based at the mountain hut at Kal on 2nd
October 2010. The terminal rift was enlarged to gain a
\textasciitilde 20 m pitch and a larger, undescened (due to lack of
rope), pitch.

A return trip three weeks later descended the pitch and found it to be
\(\approx\) 60 m. The cave closes immediately, with a tight rift taking
the water and a slightly larger abandoned rift also offering potential.
It draughts strongly.

The \emph{M2} cavers returned in thick fog, following their footsteps
through the 10 cm deep snow. With the coming winter Migovec is
effectively closed for exploration until summer 2011.
