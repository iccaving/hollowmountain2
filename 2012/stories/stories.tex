

\section{Contributed stories}


\subsection{\texorpdfstring{2012 -- \passage{Watership Down} \& Hot
Pants}{2012 -- Watership Down \& Hot Pants}}

The camp was set up and exploration was in full swing. The way out East
already had its cohort of converts. For me there was one pushing front
which had primacy above all others -- namely the deepest point in the
cave. \passage{Winter Journey}, which I had explored with Jim and Fratnik
last year, had a few niggling leads but nothing stellar. The silt
deposits and depth indicated that a sump was not far away, but the
inclined bedding heading North could continue for a very long time. With
an interest in diving the sumps, I also had a distinct interest in the
flooded sections.

Luckily Clare was easily suggested with the very bottom as a target. One
concern was that the wet pitch series through Day Dreamers had been left
rigged last year -- we had anticipated further trips after our last one,
but weather put paid to this. So we took a tackle sac of string to patch
the pitches where necessary.

The one negative point was my little Canon `pocket camera', hauled with
nary a care in a Pelicase through all kinds of horrific caving locations
over the previous six years had finally given up the ghost -- and so we
had no way to photographically record where we were visiting.

A smooth trip down to Red Cow was had, just enough of the route
memorised from last year to make it smooth. We discussed the camping
potential at \passage{Red Cow} (very pleasant, I think) and despaired at the (lack
of) quality in the rigging on the many little climbs and occasional
pitches from the bottom of Big Rock. ``On behalf of the entire 2003 \&
2004 expedition I apologise'' spoke Tetley on a previous trip with
Clare.

I was interested in checking out the `downstream' sump at \passage{Red Cow}, which
takes the majority of the water from the \passage{Republica} chamber. So we
followed a few cascades and reached a 3 metre drop with a single bolt. I
believe the survey starts around here, but it's difficult to say as this
region is poorly PSS. We attached one of our ropes, I abseiled down
managing to walk back in the chamber to avoid the waterfall. The lake at
the bottom of this chamber was not in fact the sump, rather a metre wide
phreatic tube led off for perhaps ten metres, finally reaching a bend
where the roof continued dipping but the water stayed still and level,
with pools of brown silt sitting in the otherwise pure white floor.

Derigged, we climbed up in the rift at the previous cascade, and found
an obvious dry level. This was strange, as I knew this passage was not
on the survey, but there were clear marks from cavers passing. Clare and
I decided to give it a proper push. Soon we found ourselves discarding
bits of SRT kit and harness. We made a good team, Clare wriggling off
through the smallest of gaps while I continued to expand them to human
size behind. A flat out squeeze over cobbles took a while to reengineer,
and a ninety degree bend to stand up in a wriggle rift took a lot of
hammering by Clare to make it passable. Alas, she found herself in a
region, again, with obvious caver marks the far side of the tight stuff,
and soon ``I've been here before'' as she found a \passage{Republica PSS}. We had
connected between the downstream Red Cow sump and the head of
\passage{Insomnia} pitch (climb back out across the pitch along the large
ledges, before turning right into a chamber leading off), connecting the
`crescent' shown previously on the survey. This only further complicates
our understanding of the hydrology here -- was this the old route for
the \passage{Red Cow} water, did it join up at \passage{Insomnia} to form that
impressive pitch, before it found a way out through the present cascades
and sump? Still, we had fifty-five metres of survey in the book, a good
thirty of which were new, where many cavers must have stood but none
decided to push. This also offers a tight, though not horrifically so,
flood safe bypass for \passage{Republica}. With further enlargement it could even
become the through way.

So, after a couple of hours diversion, we continued to our main target
-- the bottom. \passage{Insomnia}'s rope was badly hung up, I had to
reverse prussic about 20 m directly through the water before I'd
stretched enough slack to rig my descender and very very gently
(checking the rope) descended \& unhooked the rope from the crack it had
been wedged in.

The wet ropes in \passage{Insomnia} were found to be in good nick, and so
we abandoned the rigging gear to speed our progress. Interestingly, the
only place where the rope had been abraded was the natural tied off bit
to help you avoid landing in the big pool. Here the rope had been
dangling a natural `L' shape, and being gently swung back and forwards
by the draught had sliced through the sheath and most of the core where
it touched the rock ridges on the floor.

Back at \passage{Winter Journey}, we threw ourselves into the rift and soon
reached the exploration end. I essentially pushed Clare into the rift
taking the draught, and she started enlarging with the bolting hammer
and squeezing away. Progress was pretty quick, but I was captivated by
the inclined bedding plane leading off. By myself, I hadn't dared climb
down this last year. It didn't look so bad now, and Clare would
definitely be able to follow me into this `lobster pot' if that's what
it turned out to be.

So I slithered down and away, a pretty long way down over dried silt,
and reached a crawl way at the bottom. There was a gentle draught once
more. I called Clare down, though she had nearly made it through her
upward squeeze. We padded off through the rabbit warren like passage,
dull thuds of our paws on the soft floor. ``\passage{Watership Down}'' was
the obvious name for this find -- especially as I knew Clare liked the
book. A branch to the right led upwards to a pitch into a chamber.
Continuing ahead entered a slightly more confined space and a crawlway
climbing back up the bedding to reach a T-junction. Here a large draught
blew across, coming from a crawl / stoop leading down from the right.
The way to the left soon turned into an inclined rift climb, steeper
than other bits of bedding had seemed. But it was negotiatiable with
care. The next climb looked rather more committing, the walls clearly
belling out into a proper chamber, and what was that dark space down
there?

I must admit exploration fever had rather got me at this point, as I
climbed down without too much concern for the return journey. I was
stunned by what I climbed down towards -- a large, crystal clear, sump
with a clear rock arch leading off into turquoise blue depths. The
slight seriousness of our situation was rather underlined by Clare
arriving, with a clatter and a woosh as she lost her footing and nearly,
in that caving cliché, fell into the terminal sump.

We ignored the inclined cliff that was our only escape, and continued
exploration. The main sump was beautiful, but a parallel route through
the rocks led to a smaller, obviously connected, body of water. From
below, a clear balcony was visible above the sump, and with a rope,
would be a passable pendulum. Feeding into the sump was a dry inlet,
which led away through a small crawlway. This we pushed until it started
to get rather catchy and surveyed out, though further progress is
certainly possible. Our survey down here complete, including a plastic
PSS at sump level on the sandy beach, we had to tackle the climb. Clare
went up first, after a few false starts and slithering back down, we
found a working method utilising `combined tactics', where I would
bridge across the floor and ceiling and be stood on. Her safely up, I
slipped back down again and was lowered the end of the survey tape to
complete the measurements -- 9.83 m to safety.

There was nothing Clare could do to help as I slipped and struggled. The
mud on the walls had been made slimy, the footholds were degenerating. I
slowly slithered up the far end of the rift where it was narrow and I
could bridge most effectively, but therefore had to deal with
overhanging sections which I scaled, somehow, with the minimum of poise
and grace. As we surveyed out Clare had to indicate the stations again
and again, the buzzing in my head was dissolving memories as fast as
they were forming.

Time was pressing on and we did not have time to look at the many leads
left in the rabbit warrens. Instead we made a speedy exit, Clare taking
the exploration bits and bobs, I choosing to pull up the ropes as there
is never a guarantee of return at these extreme ends of the cave system.

\passage{Red Cow} offered it's usual calming influence; the strange `quiet
corridor in an alien spaceship' feel to these horizontal sections was
comforting where it had been disquieting. We sat there munching the food
we had stashed on the way in, rehydrating even as our Meander oversuits
steamed off their splashes from \passage{Insomnia} and \passage{Republica}. I thought
dark thoughts about the riggers of both these pitches, in their
different `exploration rigging' ways my two most hated and, in my
consideration, dangerous pitches of the cave. And there they were
together, back to back, a gauntlet to be passed on the way to the
depths, and a horrible back-of-mind barrier to the exit to safety.

The way back to camp passed smoothly. Back for tea and medals, talk of
daring do and plans for a rather more sedate second day.


\subsection{The Day After}

Our plan for the second day was rather more constructivist in intent.
Since exploring the \passage{Throne Room} last year, no one had returned
except to the large obvious windows which had formed `\passage{Amazing
Grace}' and the way to the East. Clare and I knew there to be a
potential traverse, and also a short pitch. This area is relatively
close to Camp \passage{X-Ray} and so would be a good place for less
experienced teams to cut their teeth. So we packed up the Uneo drill and
made our way along \passage{Kamikaze} and the \passage{Red Baron} traverse. The leads are as
we remember them, and I quickly rig up the start of the traverse and
place a bolt in the large boulder for abseil (with tackle sac rope
protector) into the pit. Confirming it's a going lead with a pitch
leading off, I come back up and get stuck into the main metal of the
traverse.

{[}Nb: If it doesn't get written up elsewhere, Nico et al. Dropped this
pitch, and a few more climbs and cascades before it degenerated -- from
the survey and form of the cave, it looks as if you're descending the
immature vadose formation below the \passage{Throne Room} boulder
collapse.{]}

The first few bolts are just fine -- swing and bolt, rejiggle the
rigging and move outwards one metre. From my perch around the corner, I
realise that this is rather more demanding than had been hoped -- not a
traverse over the pit and into a side passage, but a continuing climb
traverse out of the end of the chamber. I keep at it, finding passable
rock to bolt in the overhung ceiling, and somehow climbing backwards
over stopped boulders and chunks of dried mud. I am not, as it were,
enjoying myself at this point. The bolts aren't being tested as I pass,
and I'm climbing a good few metres in height between placements, on semi
static rope and without a belayer, just judging the length of slack.
Considerable quantities of footholds disappear bouncing down the slope,
flying into the pit at the bottom. I explain this predicament to Clare.
She replies with ``Well, you don't have any choice -- we don't have the
dynamic rope with us and we're not going back to camp.'' Charmed.

Clare, shivering on a boulder while I sweated with outstretched drill,
also has an idea for the survey name -- \passage{Hot Pants}. Why? ``I have a plan,
and it's as hot as my pants\ldots{}'' - Lord Flashart

At the top I reach a little chamber with an obvious climb leading up
through a boulder choke. The traverse rope is just long enough to belay
as far as the top of the climb, but not to protect the way across.
Feeling rather flushed and vertigous at the achievement (19.44 m at 50
degrees says the survey, between here and the last traverse bolt) I
finalise the rigging. Clare follows me on the rigged line, I watch it
scratch at a particularly large pile of boulders which I didn't dare
disturb -- derigging from the bottom and gardening back from the top
would make a lot of sense long term.

We climbed up carefully through the boulder choke and were rewarded with
a beautiful little chamber, about 12 x 4 m, with a high ceiling and an
obvious balcony a good four metres off the floor. Out of gear, we looked
at the crawlway heading NNW out of the chamber. This was taking a
distinct draught, and was easy (though rather small) going with a white
silt floor. Clare slithered over a little dam and wriggled off, coming
back to state it was going but rather small. I couldn't be bothered with
such arduous squeezing at this stage, and expecting that this lead would
be looked at first before bolt climbing plans, was happy to survey out.

Strangely, this crawl wasn't again looked at -- instead the teams who
came after us hand bolt climbed to the balcony and beyond, gaining
passage 7 metres above the chamber floor, which led to 73 m of gently
descending passage to the NNE (Peepshow), terminating at a boulder
choke, and then 78 m of gently ascending passage due North into blank
mountain (\passage{Undercover Squirrel}) terminating at a 4 m draughting climb
into a chamber.

And that was the story of my first camping trip with Clare. I certainly
felt rather pleased -- we had gone after the lesser leads and multiplied
them through our efforts. Finding the beautiful \passage{Watership Down}
sump was a joy, but the \passage{Hot Pants} traverse climb was perhaps the most
important work in terms of the exploration it enabled.

\name{Jarvist Moore Frost}


\subsection{The Piss Bandits and the Undercover
Squirrel}

\#\#\#weeks 2 and 3 - 2012

The prospect of spending a second summer camped on Migovec carried with
it an entirely different mix of emotions than those preceding my first
year. Gone were the fears of the unknown, replaced with an exciting
familiarity that may be felt when meeting an old friend. Perhaps more
importantly though, was a new sense of independence and confidence that
had been lacking throughout my first year. No longer was there a
reliance on someone to show me the ropes and, as such, I found myself
fitting into the carries and rigging that typify the beginning of the
expedition, feeling like an experienced member of the team. Bivi life
was great.

My sleeping arrangements were shared with Nico, an arrangement that
benefitted the rest of the expedition, as our messiness and stench was
confined to a smaller area of the mountain. Our general hygiene levels
and lack of experience lead to us christening our team as the `piss
bandits.' As we seemed to get on well, his laid-back style balancing my
seriousness and as we were both looking for that `next level of
expedition independence' we decided to form a two man team to head down
to camp and have a taste of pushing.

The focus of our first pushing trip was to be an area past the Red Baron
chamber named Queen's Road. We had been told of a short pitch with a
possible continuation at the bottom. Simple, immediate access to virgin
cave; perfect.

Needless to say, our lack of experience made the whole process take much
longer than we had initially thought. A dodgy bolt was placed by Nico
before we swapped rigging duties and I placed a suspect deviation that
allowed access to the bottom of the pitch. Immediate gratification was
not found as the next section, despite being a small drop, required a
bolt to be placed in an awkward, narrow position. Cussing ensued as Nico
placed the bolt an wriggled his was down. Thankfully, the space seemed
to open and a further 5 m pitch promised open walking passage at the
bottom.

Now feeling much more relaxed bolting, I hurried down the next pitch to
find a large boulder filled chamber. We shouted to each other with
excitement and continued down the chamber to find a disappointing
boulder choke that barred the way on. Being the `piss bandits,' we
marked our territory before deciding it was time to head out, surveying
along the way. On our way out, we checked another continuation further
along the Queen's Road passage (up Hot Pants) that seemed very promising
if bolt climbed and made a note to tell the others.

The next few days camping were awful as the wind picked up on \passage{Migovec}
and I found myself unable to sleep without being hit repeatedly by the
side of our collapsing tent. Needless to say, we headed back down having
had little sleep, eager to escape the miserable weather. During our
absence underground, Oli and Thara had attempted to complete the bolt
climb and had unfortunately run out of time before completing it.

Myself and Nico hurried up \passage{hot pants} to the climb and I set up off the
rope. Placing a couple of bolts to traverse left to a ledge I was faced
with a steep step followed by a muddy slope to the top of the pitch.
Having tried to a place a bolt in the friable, muddy rock I gave up and
clambered up the step and the slope, feeling confident that the climbing
skills I had acquired in London bouldering gyms were enough to ensure
safety. In hindsight they probably weren't as I awkwardly fell up the
slope like a new born calf just on the edge of physical control.
Standing at the top, adrenalised, the passage opened out into a large
walking passage similar. There were two extensions, one upslope and one
downslope and both seemed extremely promising. As I placed a bolt,
myself and Nico yelped to each other, discussing what we may find. This
was it, this was our first taste of promising lead!

First we headed downslope into a large boulder filled chamber that Oli
and Thara had asked us to christen `\passage{Peep Show}.' It was as the
piss bandits had hoped, easy pushing for a few hundred metres before the
passage ended in a boulder choke, just like our previous `lead'.
Nevertheless, the boulders here were much larger and the chamber floor
was built like a rabbit warren. The wind drafted in a small hole at the
limit of the chamber that Nico attempted to squeeze into. The
continuation narrowed within a few metres to the point where Nico could
not continue and, with a lot of writhing and whimpering, he navigated
his way back out. Being of the larger build, I didn't stand a chance.
Needless to say, we were gutted, however, this time we had a plan B.

The upslope continuation, that was later named \passage{Undercover Squirrel}, was
much more fun. The passage was similar to that of \passage{hot pants} and, whilst
mostly walking passage, there were a few muddy slopes and a couple of
climbs that made the passage much more sporting. One, in particular,
required us to wedge our body against the wall and slide down a muddy
slope, just in control until a rocky step was reached that could then be
clambered down. On the way back, such climbs proved more worrying. We
continued upwards until we reached a chimney that could be climbed up
into a large mud bedding plane with another climb up to a small window.
Unfortunately, another bolt climb was required and we surveyed our way
back out, pleased with our efforts. Whilst there was the absence of a
large, billowing draft, the lead nonetheless continued into blank
mountain.

Back at underground camp, a rather cold and miserable Tetley and Rhys
arrived bringing tales of poor weather up top. Whereas during my first
year underground camp had been a cold, remote area that had been
psychologically troubling to spend a great deal of time in, it had since
transformed into a place of considerable comfort and warmth; a second
home. As such, it was with displeasure that we pried ourselves away from
the soupy, cheesy, fishy, noodley smash and headed out from camp
\passage{X-ray} to the surface.

Emerging at the top of \passage{Migovec}, we found the mountain deserted. As the
weather had worsened, the rest of the expedition had fled the mountain
for a mid-Expo break. The site wasn't exactly welcoming and, with the
sun beginning to set, we made the decision to jog down to Tolmin. We
hurried down in the night, lighting the paths with our head torches,
fuelled on by prospect of a cold beer in warm, civilised abodes. We
passed the other cavers as they sat at a restaurant near the \passage{Devil's
Bridge} (Nico, ``Heeeyyy maaannn, is that the ootthhheers?''), yet,
thinking that we were hallucinating we rushed passed down to Tolmin. As
such, we arrived at Tetley's house, locked out. We collapsed in two
content heaps and waited for others, considering the improbability that
6 hours previously, we had been 600 metres underground.

\name{Jonathon Hardman}


\subsection{Diving Dreams}

A few more carries and a visit to the fleshpots and Wifi of Tolmin, and
I needed to decide what to do next. My big hope was diving, I had
stuffed a barrel full of my cave diving gear and hidden my fully pumped
7 Litre cylinders in the back of the van. Now with such a clear and
accessible (though very remote) sump, diving was a genuine possibility.
At first I decided the logistics were simply impossible, but there
seemed to be a lot of people making positive noises about portering
assistance, and so, deciding to commit, readied my gear at \passage{Ravne}.

Some of the Slovenians had promised help in portering the gear to \passage{Kal},
unfortunately the motorbike broke terminally while carrying the
cylinders. Abandoned in the forest, they were carried the last zig zags
by hand. My lead weights arrived in the \passage{Bivi} almost as if by magic.

But every move closer to the pushing front had fewer and fewer
volunteers. Offers of assistance are gladly made in the sunny cafes of
Tolmin. Fundamentally when the choice is porterage of diving gear at the
expense of dry exploration, the choice of expedition members was clear.

Ideally the diving also required the setting up of a camp at \passage{Red Cow}. I
had prepared a lightweight two-man camp for this application, and had
assumed that people would at least be interested in visiting the bottom
of the cave or extending the leads further North in the deep levels.
However, one can't really direct exploration in this way, and the
majority of the expedition efforts went into the extensive horizontal
levels found below \passage{Stuck in Paradise} -- only two trips went
beyond \passage{Big Rock} this summer, and I was on both of them.

I solo'd my way to \passage{X-Ray} with a 7 Litre bottle and fin in a
tacklesac, and a weight belt of 8 kg of lead. This was actually found to
be not too difficult, though frightening on the little down climbs!

Approaching the end of my three weeks on the mountain, I finally had to
accept that the sumps were out of my reach. I was exhausted through my
efforts, making myself sick through not enough rest.

My last mad plan involving hijacking the ever eager Oli and spending
essentially three days portering gear to the dive site, diving, and
abandoning cylinders for the winter, was replaced by a rather more sane
plan to explore the bottom, and, sadly, drag my cylinder out from camp.

I pulled my shoulder really quite badly during this trip, which made my
slow exit from \passage{X-Ray} hauling steel arduous and painful. The
situation on the surface wasn't much more pleasant. The \passage{Bivi} talk was
full of suspicion that I would abandon without carrying all the dive
gear (and, indeed, my unused photographic gear) down. So there was
nothing to do but swipe opiates from the first aid kits, ignore my
grinding bones and do my carrying duty. I must admit it was with
considerable sadness and little will to return that I finally struck my
tent, squeezed the last few bits into my rucksack, ached down the
mountain and boarded a bus out of Tolmin.

My 8 kg of diving Lead sits at \passage{X-Ray} currently labelled with a
simple note:- ´Jarv's Folly¡

Everything is clear in hindsight. I should never have started preparing
to dive, instead put my efforts into dry exploration \& bolt climbing of
the deep levels and, potentially, readying the ground for a diving trip
in future years. Still, we live and learn, and ´Ah, but a man's reach
should exceed his grasp, or what's a Heaven for?¡

\name{Jarvist Moore Frost}


\subsection{\texorpdfstring{Cutting your neck at the
\passage{Guillotine}}{Cutting your neck at the Guillotine}}

It is always a long time that passes between two expeditions. A time for
living the normal life, doing the work, caving a lot, fixing or
purchasing new gear, and -- making plans for the next year.

As so much passage had been discovered within the -600 m phreatic level,
it became reasonable to hope that there might be a connection between
\passage{Sistem Migovec} and \passage{Gardener's World} at this depth. The
efforts for finding the connection at -400 had been failing for years,
despite the plenty of action and enormous energy that went into this
action. Meanwhile, large amounts of passages had been found at -600 with
much less effort, and sometimes they went as far as 500 m on a trip.
Therefore, it was not insane to look for the connection here.

But where could it be? Although \passage{Vrtnarija} had now several
kilometres of passages at this level, the System only intersected it.
Back in the years the primary goal was to reach the -1000 m barrier, and
it was evident looking at survey -- there were almost no horizontal
sections that had been found below 400 metres depth. There was only one
place on the survey which resembled something like \passage{Friendship
Gallery}: the area around \passage{Elephant} and \passage{Waterloo}, not far
from \passage{Hotel Tolminka}. To make things even more complicated, these
could not be reached without changing the ropes, and we did not have the
manpower and the gear to do so. Therefore, if one wanted to find a
connection, it had to be done from the \passage{Vrtnarija} side. Given the
dimensions of the cave, this was nothing less than finding a needle in
the haystack.

Looking at the plan of the cave, we noticed that \passage{Minotaur rift}
was heading more or less towards \passage{Elephant}, about the same depth
level. The horizontal distance between these two points was about 250 m,
which was not so much considering the potential in the phreatic level.
Moreover, the \passage{Minotaur} rift was formed along a huge, very well
defined fault line, which had a strong potential for continuation at
both ends. (This fault could also be found at the -800 level). It is
there that I wanted to look for the passage leading to the System!

After the great pushing day with Kate, Clare and Rhys and finding
\passage{Atlantis}, we decided to go there with Rhys. At the end of the
\passage{Minotaur} rift, just where the crystals appear and the passage
turns towards the right, it was possible to continue along the fault
line in a small passage. We only had to continue this line, and it was a
straightforward task to do. This has already been pushed to a boulder
choke by James Kirkpatrick. There was only one minor problem: the fault
line itself! For obvious reasons, the rock there was extremely loose and
broken. So, on one hand, we had to move a lot of rocks in order to make
progress, but on the other hand, we were never quite sure whether the
ceiling was going to collapse on us or not.

There was one point along the passage, where we needed to squeeze down
on the left hand side. We managed to move quite a number of rocks, but
one large one remained there, and there was nothing for it but to
squeeze underneath. When I finally attempted to pass under this rock
blade, of course it moved\ldots{} but luckily, by not so much. Since
there was no space for putting the rock anywhere else, we tried to stack
smaller rocks underneath in order to keep it up. Still, when you
climbed, your head was directly under it\ldots{} good luck! So, the name
of the passage became \passage{Guillotine}!

Eventually, after long hours of battling against the rocks, we managed
to emerge in an open space: the fault line opened up again! We found
ourselves at the top of a \textasciitilde 30 m high rift, with an
active streamway, with beautiful white walls! In the squeeze, we found
haematite pebbles and in some places along the fault line, the rock
looks like marble, which would be very nice were it not for the fact
that it constantly rains down on your head. Crawling out from
\passage{Guillotine} was even harder than going in, since the whole passage
was sloping downwards. Moreover, as we crawled, the oversuits became
packed with sharp little rocks, while other rocks constantly fell on our
heads. A truly fascinating place! Still, the continuation was at the
bottom of a large, open and going rift -- quite the lead\ldots{}

Later, we returned with Jana to rig the rift at the end, and to survey
the passage. The end of the rift needed some bolt climbing, but we could
hear that it continued on with large volumes. Moreover, to our surprise,
the passage went more than 100 metres in the direction of the System!
Thus, the distance to the connection became less than 120 metres. We
then planned to go back there to climb up the following year. Thanks
God, the events that followed later made this heroic attempt unnecessary
(since \passage{Sanje za Duso} \sidenote{\passage{Dreams for the Soul}}
crosses the continuation of the rift). It is thus likely that nobody
will ever enjoy the special treatment of \passage{Guillotine} again. Who
knows, maybe since then, the large boulder has already fallen down,
sealing the passage for eternity\ldots{}

\name{Gergely Ambrus}


\subsection{2012 -- Watership Up with Oli}

Oli hadn't yet been on the serious pushing trip he was clearly capable
of. Having abandoned diving plans, I went down with him on the standard
two nighter. First stop, before bed, was the fabled \passage{Esoterica}. This was
quite an interesting place, being a streamway cutting through the \passage{Prince
Consort Road} horizontal level before the \passage{Albert Hall}. The \passage{Serpentine}
stream at the \passage{Albert Hall} almost certainly forms the development which
goes all the way to \passage{Watership Down}, there's every reason to
believe this streamway may proceed similarly. However, tales of wet rift
from the original explorers (James KP and Jan in 2010) have dissuaded a
return. We found the climb down through the boulders, and arrived at the
10 m pitch (taking water from the little inlet in the left of \passage{Prince
Consort Road}) I'd previously stood at. We commenced bolting, taking
turns.

While Oli was tapping away, I climbed over the rift pitch, down a 4
metre free climb and intersected a different stream. This stream came
down (on the left) from an easily gained chamber, with vast quantity of
beautiful haemetite on the shelves. The continuation (to the right \&
following the stream down) was obvious, and I can only assume that this
is \passage{Esoterica} as pushed by James KP and Jan.

I related this to Oli back at the pitch head. We finished the backup
bolt no problem, but horror of horrors, broke the driver placing the
hang bolt. A few minutes experimenting with alternative natural rigging
(all too dodge) and we accepted unhappy defeat. Should have brought a
second driver. Should have brought the drill. I left a long and
meandering note underlining the point that this pitch was not dropped,
and the way to what I believe is \passage{Esoterica}.

It's one of those strange quirks of expedition -- we had what seemed to
be two independent streamways, both going and barely an hour away from
camp. Yet no one returned here either.

Having decided returning to camp for a replacement driver was too long a
round trip, I took Oli on a tourist trip to Palace of \passage{King Minos}.
We admired both the formations, the unpleasant nature of
`\passage{Guillotine}' and the admirable efforts in bolt climbing the
\passage{Queen's Bedchamber}, and returned via the \passage{Oroborus} alternative to
the \passage{Albert Hall} -- odd passage formation indeed.


\subsection{Day 2}

We set off for the deep. I find \passage{Big Rock} a rather unfriendly place.
Though no one had been here since my previous trip (in 2011), I found a
lot of the bolts disturbingly loose. Always a worry with rawl bolts; I
also had dark thoughts of some kind of bolt loosening cave monster,
which I imagined as some kind of spanner assisted sloth.

The way to \passage{Red Cow} was almost second nature now -- the third time in two
years. I pointed out the few bits I'd figured out about the route on the
way, and dealt the stack of playing cards I had filled my Meander pocket
with (arrows and `\passage{Red Cow}' or `\passage{Big Rock}' to waymark the two opposite
destinations). At \passage{Red Cow}, we repacked and left a Daren drum and food
stash, then continued on down to the bottom with a single bag of bolting
kit, rope and survey gear. Oli made short work of the sideways wriggles
in \passage{Winter Journey}, and with sanity rather than wanderlust on our
side we set about rigging the steep muddy climbs which Clare \& I had
slithered down and had such trouble climbing. While I merrily tapped in
the first bolt, Oli had a quick look at the `upwind' route from the
T-junction and came back shortly having wandered to the bottom of a
climb he didn't fancy by himself.

At the main climb, I realised I'd rigged the scrappy ropes we'd brought
in the wrong length order -- Oli sorted this out behind me while I
placed a Y-hang off the opposite walls. The depth of the dry silt was
impressive -- I chiseled off 5cm deep plates from both walls to find the
rock.

Arriving at the sump with rather more poise and grace than last time, we
noted that the plastic PSS at the shore edge had floated off into the
sump -- clearly the level had increased (at least minutely) at some
point during the previous two weeks. I also threw a few rocks in the
sump to try and gauge how deep it was (4 m I estimate) and stomped in
the shallows a bit to try and assess the visibility for diving potential
(remarkably good, surprisingly no brown mud, the coarse sand settling
very quickly). Wearing my wet socks, I was trying to psyche myself up
into traversing along the right hand (sloping rock ledge) of the sump to
try and assess whether it continued around the corner, or stopped with a
rock wall.

Oli reckoned he could have a chance at the balcony, and prussic'd up the
rope before swinging across. He left the rope pulled over, but having
got together our survey instruments I grew bored of waiting for him to
return, and flicked the rope off to follow.

The level gained, \passage{Watership Up}, was really quite interesting -- very
difficult to tell how it formed, other than the obvious bedding plane
bit and with strange (large!) echoes leading off from tight rifts.
Following the natural route past puddles of Hematite and along a
protracted flat out squeeze we came to a section of rift with a drop
down onto a lake. This was a definite lake, as we could see both ends of
the body of water, but otherwise looked very similar to the main sump.
The pitchhead down from the rift would be narrow but certainly doable.
We surveyed our way back, notably there was a collection of those
strange spider-web like filaments covered with droplets in a cranny of
the rock. Always a bit spooky when you come across them -- no matter how
much one reminds oneself that they must be some kind of slime mold or
fungus, the image of a massive cave spider is hard to shake!

With time already surprisingly tight, and a long way to climb, we set
off out, pulling all the ropes up with a bittersweet melancholy. We had
put the rigging effort in to further the next party, and effortlessly
found passage, but there remained so many question marks, and no other
parties intended to return down here this year.

Back at \passage{Red cow} we toasted our mackerel with a splash of the meths
recovered from `\passage{daydreamers}', and then had a warm Vitaminski (pusher's
delight). We filled up our Daren drum with water -- measuring the \passage{Red
Cow} stream as 1L/s (by filling a 6L Daren drum in a timed 10s, capturing
about 60\% of the stream). This Daren, along with the `bivi liberated'
mess tin and blue 2L Sigg bottle (with perhaps 200 ml of meths left) was
left sitting in the sand of the main \passage{red cow} passage. There was talk of
bolt climbers making their way down this way to investigate `\passage{Strap on
the Nitro}'. This didn't happen, and so those resources stand there
currently, awaiting the 2013 developments.

The slog on home to \passage{X-Ray} continued with little comment -- other
than I managed to pull a shoulder on the \passage{Memory Lane} climbs. We also
managed to flood the passage with smoke by leaving the fish tin to burn
itself out (also burning the paint). Interestingly, and in contradiction
to the original explorers memories, the cave draft took this in the
direction of \passage{Big Rock}, we found ourselves caving through this fug till
at least \passage{Memory Lane}.

The way out from \passage{X-Ray}, with diving cylinder in tow, was slow
indeed. I sent Oli off ahead, and laboured my way behind. Some of the
\passage{Urinal series} pitch heads were really quite acrobatic when dragging a
third leg through!

\name{Jarvist Moore Frost}


\subsection{\texorpdfstring{\passage{Sanje za Duso} (\passage{Dreams for the
Soul})}{Sanje za Duso (Dreams for the Soul)}}

The last pushing trip of the expo is always a bit extreme: we don't have
to reserve our energy anymore, so let's go with full power! It was
really great and exciting news that Mafi finished the climb of the
\textbf{Apollo} in \passage{Queen's Bedchamber}, and at the top, yet
another phreatic passage was found, with some junctions that they did
not have time for checking out. So it was an obvious plan to go there.
In fact, at the top of the climb there were two possible continuations,
and the other one (at your back when you climb up) seemed to be even
bigger. So, my plan was to traverse across the top of the pitch to reach
this possibly large gallery. I managed to persuade Karin to give it a
go, and there we were, going down on the well-known route to
\passage{X-Ray}. Here we come, \passage{Vrtnarija}!

Of course, we were dreaming of the connection, or rather joking about
it. Karin said that she had a good name in her mind, which was also the
name of a hard climbing route that had been completed recently. But when
I asked her what it was, she replied that she can only say it if we find
the connection. So there we were, with a (hopefully) good name for a
passage that will only be given if it connects the two systems\ldots{}

After a good night sleep, and a healthy morning meal with meat and
tomato, we packed up all the gear for climbing the traverse: drill,
bolts, hangers, maillons, rope, and so on. Reaching the \passage{Queen's
Bedchamber} is a pleasant trip. There is one peculiarity though: at the
low passage in the chamber with the plenty of crystals, again there was
almost no wind. In the first year, when we found it, there had been a
huge draught here. Probably it was connected with the excess of water in
that time: almost all the big pitches had large waterfalls in them.
Nevertheless, we were certain that something big awaits us at the top of
the \passage{Apollo} climb.

I made my way up carefully on the rope. It was absolutely terrible. The
pitch was (and remains) the worst in the whole system; one can see the
heroic effort to reach the top, climbing through mud, montmilch, loose
rocks, and so on. Hopefully, a bypass will be rigged one day
here\ldots{}

Anyway, both of us managed to get to the top without killing ourselves
or the other, and we started the preparation for the climb. However,
just after I drilled the first hole, on cable of the drill came out
loose! There was nothing to do with it, so there we were, with all the
equipment, but without a working drill\ldots{} The climb was way too
long to be done with hand bolting. So what should we do? Well, let's
check out that junction that Mafi was talking about (and for which,
jokingly, I said that there is the connection :)

(The next year we did
the climb with Peter\sidenote{likely Peter Adamko, Gergely's friend and a fellow Hungarian}. Guess what? There is nothing on the other side!)

The passage leading there was nice, and soon we found ourselves at the
junction. Entering the passage, our good old friends showed up on the
wall: the wind-crystals! Hmm, so there used to be quite a strong draught
here\ldots{} let's see where it comes from!

We had to work on two squeezes for quite a while, especially that we did
not have a chisel, just a bolting hammer. But the crystals were always
there, showing us the right direction. (There was also a strange thin
material, like spider web or hair, in some places. Maybe that is formed
by bacteria, but it should be checked out\ldots{}) Moreover, as we
proceeded, we constantly checked the compass, and it seemed that we are
heading directly towards the System! The distance that we had to cover
was about 150 metres.

The passage was small, at one squeeze I had to take off my SRT to pass
through. Finally, after negotiating yet another squeeze, we entered in a
big space! It was a rift, about 25 m high, and checking the direction,
we were certain that we managed to get to the continuation of the
\passage{Minotaur} rift-\passage{Guillotine} fault line. So, the game was on!
We climbed up in the rift towards \passage{Guillotine} in order to have a
look, but that was not the goal why we were here\ldots{}

Just as in the fairy tales, a nice open passage started from basically
the point where we emerged in the rift. Its direction was right towards
the system! Our distance at this point was about 50 metres. There was a
draught and crystals, so we became quite excited\ldots{} We started to
follow the passage. Then a climb up, then\ldots{} it is blocked here!
Hmm, a dead end. But going back to the passage, we noted that it
actually has more levels, it was a phreatic meander. We could pass one
level below, then again up, then again down, then\ldots{} we ended up in
a passage which lead to the side of a fairly big pitch with a
waterfall!! By that time, I knew the description of \passage{Waterloo} by
heart, so I became very curious. There was a small ledge, about 30 cm
wide, on the right hand side of the pitch, going across (maybe 5 m
long). Luckily, we had some rope and maillons, so I started off from a
natural, and proceeded carefully along the ledge. Then, in the middle,
where I had to step down, I saw\ldots{} 2 rusty spits!! So, there it
was, there we were: the result of so much effort by so many people, on
the last pushing day of the expedition, following the wind crystals, we
got into the system!

I called out to Karin. ``Hey Karin, what is the name that you had in your mind?''

``What?''

''Well, you better remember the name, because
there are two rusty spits here!''

At the beginning, she did not want to
believe it, but after I got to the other side and fixed the rope, she
saw it herself. We stood for a minute, still not believing what
happened. On the other side of the pitch, there is a nice big chamber
(\passage{Waterloo}), and soon we found the PSS 13 from Dave Wilson and
Andy Jurd, who have been here in 1998. We built a huge cairn, and placed
a sign, stating the exact date and time, and the fact that \passage{System
Migovec} became the longest cave system in Slovenia! And the name?
\passage{Sanje za Duso}, which means \passage{Dreams for the Soul}\ldots{}
really an aptly named passage!

Fate is a strange thing. If back in 1998, instead of descending down
\passage{Waterloo}, they had traversed the pitch and followed the passage
that we came through, they could have found basically the whole
horizontal development in \passage{Vrtnarija}. On the other hand, we would
probably not know \passage{Concorde}, \passage{Pico}, or \passage{Happy Monday}.
And also, we could not have found the connection, and could not have
this story either.

What happened after? We carefully surveyed everything of course.
Descending down the \passage{Apollo} climb was just as bad as climbing up.
Back in \passage{King Minos} palace, we admired the crystals one more time,
now knowing for sure that indeed, there was something big behind them --
namely, a system of at least 12 km of caves! Back at home, in
\passage{X-Ray} (which was 70 metres lower now than in the morning), we
decided to celebrate by a candlelit dinner, and by finishing off our
whisky reserves. So we made one hot chocolate whisky after the
other\ldots{} As a result, the next morning I was in terrible shape,
although Karin did not have any problem. Strong girl! We started to pack
up, enjoyed the last solitary time before the transporting team arrives,
and then the news started to spread -- the connection has been found!
Big celebration on the surface, media, party -- but that is another
story. Jonny's final entry in the logbook for 2012: ``So, an ordinary
derig was made much less ordinary when Gergely + Karin told me that the
cave is now 70 m deeper. Hooray for sys-mig.''

It was a moment that is going to stay with us for our whole life. But I
particularly like it because it really has been the result of the effort
of the whole group, over so many years. We had been lucky to have the
chance of being there -- but we had been even more lucky to belong to
such an excellent team of nice friends.

\name{Gergely Ambrus}
