\section{The Slov magic (2008)}

Shortly after arriving to the plateau, I found myself getting ready for
a trip. Not only a trip, in the ordinary sense, but a real Exploration
Trip, with a goal none less than trying to remove the large boulder
blocking the tight rift at the bottom of \passage{M2}. I was chosen to be
the lucky one who accompanies the Living Legend of Tolmin cavers. Soon I
found myself holding a nice yellow tacklesack, with the instruction of
``be gentle with it''.

And this was not without reason. In the bag there was to be found an
original memento of the great war, a product of the ever famous factory
in Glasgow, dated to 1937. A well educated friend of mine pointed out
later, that products bearing the name of a prestigious prize become
highly unstable in a period of cca 50 years, but little of this
reassuring fact had I known back then. Instead, I took the tacklesack
happily and we started our vigilant journey towards the centre of the
globe.

It soon became evident that the only way for being faster than my master
would be to use gravity directly, without the unnecessary complication
given by ropes, re-belays and so on. The main obstruction was a nasty
wedging squeeze somewhere, which proved to be even worse on the way out.

Finally, we reached the bottom, set up our magical set, murmured a
couple of magic spells, and then woops, the boulder evaporated as if it
had never been there! Indeed, it does help in exploration to read Harry
Potter after all. We started to chisel our way further, but the rift got
even narrower after a couple of meters, so we had to give up and started
prussiking up. Again, my companion was nowhere to be seen until I
emerged completely exhausted at the bottom of \passage{Silos} -- where he
waited half asleep, and noted that he is indeed quite cold. It was just
the opposite for me. The rest of the trip continued in the same manner,
and finally we emerged triumphantly at the surface. This was my first
battle with Migovec, and so to say, I got completely addicted to the
sweet smell of gunpowder.

\name{Gergely Ambrus}
