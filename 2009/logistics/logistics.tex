
\section{Logistics}

Our logistics have been heavily optimised over the last ten years of
returning to this same plateau. The main difficulty is in lifting (this
year purely through manpower) our food and equipment from Tolminske
Ravne (where we can drive to) at 912 m to the bivouac in a shakehole
under a rockbridge at 1860 m (`The Bivi').

A further optimisation that we have carried out the last few years is in
using the derig carries at the end of the previous year to bring up
sufficient non-perishable foods (rice, pasta etc.) to eat during the
first half of the expedition in the present year. This way, caving can
start fully after just two or three carries per team member, rather than
the more traditional `week of carries' that characterised the old
six-week expeditions!

A further refinement was leaving from London with the Minibus on Friday
night. Though rather harsh on the drivers, this meant that we arrived in
Tolmin on Saturday just in time for a well-earned proper meal and a full
night of sleep before an alpine start. Up shortly after dawn, we had
managed to acquire the necessary petrol for cooking and other locally
bought fresh food.

Our drill batteries were charged on mains power down in the village and
then carried up, but power for rechargeable lights, survey laptop, MP3
player \& underground camp speakers were all provided for by a small
photovoltaic tent placed next to a tent.
