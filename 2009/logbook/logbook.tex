
\section{Logbook Stories}


\subsection{Main (big red A4) logbook:}

05.08.09 James Kirkpatrick and Izi

Climbing \emph{Metal aven}

We set off for a doss day and decided to climb the 4 m aven on top of
the tent at \emph{Metal camp}. The aven leads to a roomy room (there is
a lead off a side chamber). Izi climbed up to a bolder move at that
point I scurried down and fetched rope for Izi. I secured a rope to a
sling and I joined him. The next move looked fine. So I roped up (with
the superstatic) and Izi belayed me to a few dubious flakes where the
crurx was! Crux! Exposed move 10 m off the terrace secured by a
superstatic rope and to vary dubious sling. Anyways\ldots{}..Another
mudy chamber was reached pushed to the left to a narrow muddy squeeze
(not really worth digging, I am HO) and a rift to the right which was
pushed to a constriction (lead). Which could be easily bashed --
probably the connection.

\name{James Kirkpatrick}


\subsection{\texorpdfstring{\emph{Metal Camp}
menu}{Metal Camp menu}}

Couscous, smash, fish and smash with tomatoes Smash, smash, smash,
cheese and fish Couscous, fish, tomato, couscous and cheese.

\name{Thara}


\subsection{Jana and Dan --- 63h pushing
underground}

7--10th August 2009

We were suppose to be on a night train team following Tim and Thara. But
after spending a day quite active (caving and a carry) we decided to go
down in an early morning. Set an alarm for 5 am and start caving at 6.30
am. Down in the camp we woke the day team Tim and Thara. After breakfast
and a chat we swop the BEAST comf and go back to sleep. Thara and Tim
continue pushing the lead, which is still going. They come back at 10 am
and woke us up. Our first time sleeping in a camp was quite broken. For
the first 5-6 h we didn't really sleep. We also put extra extra comf in
our sleeping bags -- it was cold!

Tim and Thara pushed the cave for 2 pitches down: 10 m and 34 m.
Afterwards there is a climb up a bolder choke where they stopped. So we
went up, where on the other side was a pitch down. We spend two hours
rigging and gardening. Basically they are rocks and bolders all around
the pitch head. Going up and down was still dangerous -- stones
constantly falling down. Needed to be re-rigged. The way on then
continue up and into another smaller bolder choke. Under there was a
small pitch down. From there we first climb up the rock and end up above
big black hole. We throw a stone down and we can hear that there wats a
long way down -- 4 seconds! Fucking hell -- that is like 80 m pitch.
Very excited we keep on throwing stones down. The echo at the bottom was
amazing. We could also hear that there was a big slope at the bottom.
From here the rigging down was not really good. So we looked for a
alternative way into a pitch. Further down there is a rift, which you
climb into it and it takes you straight to the pitch -- beautiful place
to bolt. Here we decided to survey from here back to the walk the line.
Back in the camp after around 12h of caving.

Tim and Thara finish their caving in \emph{Gardeners' World} and went
back out. Alone in the cave we had to set up an alarm. We were woken up
by Tjasa, Izi and Erik, which were on a day trip to do some climbing in
\emph{Primula}. We re-start the alarm like 3 times and on the end end up
15h in bed! We finally got up at 5 am and start caving at 7.30 am. We
were excited, finally going down the big pitch and to see how big it
really is. We took two bolting kits to speed up. I spend time re-riggin,
bolting and more gardening below Walk the line and Dan went down to
start bolting the big pitch. After 3 hours we were ready to descent. Dan
offered me to go down first. I was ready to go and looking down was just
a bit scary, plus not having practice in dealing with enormous pitches I
thought would be better if Dan goes down first. He made another bolt
approx half way down on a tiny ledge. When he come down he shouted -- O,
my god, it is really big! Quickly followed down, looking around on the
way- Amazing! At the bottom we explored around and found the way on
under the bolder choke. There is another approx 10 m pitch on. We had
some lunch and then was time to survey it. The bottom was impressive
\(20\times20\) m. Cuz the tape was not long enough we had to mark the
rope, using a zink-oxide tape twice on a way up. The length to the
rigging spot was 75+15 m to the top. And there it was - a 90 m pitch.
Pleased with our mission we had to speed up to get out on time. Our call
out was 10 pm. We stopped in a \emph{Metal camp} for hot chok and to
pack the stuff to be taken out. We were out from the bottom in 6h. Back
in the bivy at 9.45 pm.

We had a great trip and it was first time for both of us to be
underground for so long and to discover such a big pitch -- which still
does not have a name.

The next day, during the breakfast time we decides to name it:
\emph{Happy Monday}.

\name{Jana \v{C}arga}

13/08/09

Izi, Dan , Jarv @E1

Took 27 caps. Used them all. A lot were double or bust, but still
\textasciitilde 18 shot holes \textasciitilde 15cm deep, using only 1/3
of the 7.5 Ah SLA battery (24V with the Bosch). Blew a lot of rock.

Surveyed out, \textasciitilde 30 m deep.

14/08/09

Jarv, Gergely

Back with Gergely -- two more caps some hammer \& chisel action \& we
were through! Placed a rawl (stainless) for pitch.

New chamber is 23 m with 4 m climb into prior discovered stuff.

Rubble on floor -- dug for \textasciitilde 10'.

Got to bedrock.

No better nor worst than capped pitch, but not the stunning lead we were
hoping for.

13-14/08/09

Thara and Tim

Fixed \emph{Kill'em All} disloadged a large boulder down the pitch. New
bolt was put in place.

14-15 Thara and Tim Continued down Dan's lead down one more pitch --
double hammer action lead into a chamber clearly below \emph{Happy
Monday}. Hanging death a size of whole chamber kept everything from
falling down. It's like anti-gravity room where rock just floats itself
in the sky. Followed an obvious riftt to a 5 m pitch into another
chamber .flat floor with a lot water coming through just like Yorkshire
caves .walked further downstream and found spitz -- clear someone has
been here before .back to chamber Tim treid to rig a traverse line
across to another lead half finished before heading off

Thara was completely broken by the top of \emph{Happy Monday} while
Tim's bowel rumbled once again. Great find and survey.

Left: rope some bolting kit + tape +sling at the bottom of \emph{happy
monday}

hammer and chisle

at the top of \emph{Kill'em All}

PS: From final pitch, there is a traverse line to the right. It is not
great as a proper traverse, but if you descent the main pitch, you can
use the traverse line with a cows tail to reach the other side. There is
a bolt to go down from there but it needs a backup. Once rigged it would
be better to descend main rope then up the other one.

15-8-09 Jana, Gergely and Dave. Down to \emph{Zimmer} + connection

The two goals of this trip were to find the rope that Tim and Thara
rigged from \emph{Happy Monday} plus survey and rig the rift below
\emph{Zimmer}. The connection is found at \emph{Falls Road}. Climbing
down at the rift of the connection of \emph{Falls road} and
\emph{Friendship Gallery} a \textasciitilde 10 m drop reaches a Y hang
to the top of Free Amalgamation. The bottom of \emph{Happy Monday} is
about 40 m away. Below \emph{Zimmer} and active streamway is found with
beautiful lakes and a high meander. Top muddy passage of meander is
passable, but a drop down to the waterfalls would be better. Probably
the largest active known streamway on the mountain at the moment, highly
probably it leads somewhere unknown. Nice falls and waterfalls.

Super nice trip -- show-cave area!

17-08-09

Neither of us had been further in the system than \emph{Hotline} so
there was a bit of stumbling around to reach \emph{NCB}. Got to see some
of the big cave system has, that I would not see in the uk. When we
reached what we thought was the west end of \emph{NCB} the only thing we
could find that was possibly an undescended pitch described were two
large holes further back that connected at the bottom. However, there
were two holes at the bottom of that, one was descended one was not.
While James froze I descended 20 m to find very little at all and we
decided to head out. The way up was slightly marred when I sent a large
bolder down where James had been standing moments before. Otherwise we
got out in good time and slop was still lukewarm. We were told
afterwards that \emph{NCB} had a different west , so it is unclear where
we went.

FROM THE UG CAMP LOG BOOK

30/31 July 09, Andy and Jarv, The first one

Two heavy sacks from the surface, picked up 4 comf sacks at the
\emph{Traverse Chamber}. Passed threw squeezes with the aid of the handl
? Cord. \emph{Something fishy} was a bit of a campsite-only room for one
pit. Dumped sacks and continued to look for greener pasteures..and
refound here! Looked very nice at first -- a few drips, flat, dry-ish
mud. Moved rocks and set a tent -- the place not looked so good, bit of
a guagmire?. Found 1,5L of water in \emph{Something Fishy} -- drips
provided about 5 mL overnight.

SO Dangersmouse/drips on Kill em All way have to be the way
forward\ldots{}.
