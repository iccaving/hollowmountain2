\newpage
 
 \begin{tcolorbox}
 	\begin{fullwidth}
 	\chapter{Post-script}
 		\begin{multicols}{2}


            \paragraph{Editors}
            The primary editors of this publication are Javist Frost, Fiona Hartley and Ben Richards. 


            \paragraph{Editing the Hollow Mountain: 2007-201x}

            Our previous publication on Slovenia [The Hollow Mountain (1974-2006)] was a labour of love in getting it out there. The contributions were written by various JSPDT and ICCC members, but an extensive amount of time was spent rewriting, cross checking and generally hammering it into shape. The two major hurdles were that we had waited so long to finalise a publication (people from the early expeditions had moved on, memories fade; and each year of exploration makes more work for the writeup); and the choice of publishing it in a ‘scrapbook’ style with cartoons, quotes and survey excepts all mixed in with the prose.
            
            We are presently trying to get our words together to make another publication, but the hope is to achieve it with slightly more easy collaboration \& less time-intensive subediting. We’re not sure of the format we’ll choose for the final publication – A4 looks to be the obvious size to keep the costs down while still being able to present cave surveys of a vaguely useful scale.
            
            My hope is to have a fairly automated publication route, using Pandoc to convert from some lightweight markup language (such as markdown) to a customised XeLatex template for PDF production, while retaining the ability to make an Epub version for electronic book devices. Our previous book was published with a ~100mb Word document for each yeah, which became ever more fragile and prone to corruption as editing progressed. An incredibly painful process that I’m keen to avoid! 
            
            For now, the prose in fairly piecemeal form is being assembled, written and editing on Google Docs, which enables us to automatically present a webpage of the latest draft.

            \name{Jarvist Frost, 8 January 2013, \textit{Migovec Microblog}}


 			\paragraph{Collection of the articles}

 			As described above, work on this publication started in 2012, in the wake of the \textit{Sledi Vetra} expedition when the cave systems were connected. Jarvist decided that volume 2 of The Hollow Mountain would cover 2007-2012, and hoped it would be a collaborative project. Through 2013 he gathered content, proofred said content, and committed edits in Github. By mid-2014, though, one can assume he ran out of motivation to single-handedly finish the gargantuan task.
 
            The master documents for Hollow Mountain 2 sat on the ICCC Google Drive and Github for many years. Indeed, a pdf draft of the majority of this content existed and sat on one editor's hard drive for not quite ten years. The information was available, but unpolished. It awaited someone with both the motivation and the commitment to pick up where Jarvist had left off.


            \paragraph{Operation Finish HM2}
            
            It was October 2023 when Ben Richards decided he'd initiate 'Operation Finish HM2', and spent a day turning the existing content into \textit{.tex} files readable by \textit{Overleaf}, a collaborative cloud-based LaTeX editor. He was assisted by Tanguy Racine who provided the HM3 \textit{Overleaf} project as a basis. Overleaf is easier to use than Github for those who aren't used to hardcore code wrangling.

            Ben then brought Fiona Hartley onboard, and the two began choosing photographs, surveys, and small amounts of additional content for the publication. In particular the project's formatting needed altering to look like HM3 (not only quotes, recipes, songs etc, but also in manually replacing every emphasised passage name with a specific command that allows the cave passages to be indexed).

            (more detail as things progress)

            Undoubtedly we could have continued to explore Migovec without the publication of HM2, or indeed with only that initial draft from 2013, read only by those keen on diving into the backstage areas of ICCC's online storage. But not only has the lack of HM2 diminished later explorers' knowledge of the mountain's cave systems in notable ways, to leave the information hidden in the Cloud is a disservice to the ICCC and JSPDT members who dedicated their time and money to exploring these caves. They put in the work, and it deserves to be seen.


            \paragraph{Authors}

 			The authors who provided reports, anecdotes or short stories come from a wide spectrum of explorers ranging novices to longstanding members of the club, each with their own baggage of experience and perception of exploration. Restricted though it is, this variety of voices paints a fuller, more nuanced take on the story of exploration.


 			\paragraph{Format of this publication}

 			In view of their length, many of the collected articles span more than a double page. We have enhanced these entries by providing supplementary photographs, grade 1 surveys and logbook extracts which offer a nice counterpoint.


 			\paragraph{Language}

 			This is not a bilingual publication but when available we have included original reports in Slovene, and with these we have provided an English translation. The publication keeps, as such, a close focus on the doings of ICCC members over the summer expeditions enhanced by JSPDT counterpoints. 


 			\paragraph{Graphics}

 			This present volume relies heavily on colour photography to convey the scale and morphology of the cave, as well as provide snapshots of the life on the Mountain, so we must extend a huge \textit{thank you} to the photographers who lugged extra equipment inside the cave and took out the time to document the newly discovered passages, almost always at the expense of further exploration. 

 			In keeping with the previous two volumes, we have also included `\textit{Grade 1}' surveys and cartoons recorded in the various notebooks brought back to the UK. As far as possible, these have been the benchmark for drawing the cave passage outlines on the fair copy surveys. 

 			The photographers and cartoonists are credited for these contributions in the captions.


 			\paragraph{Surveys and maps}

 			\begin{quote}If it does not appear on the survey, it does not exist. 
 			\end{quote}
 			The survey making procedure is detailed in The Hollow Mountain 3, and we refer you to that chapter for the specifics (which certainly escape at least one of the editors!). It's safe to say that without the time and effort spent in drawing the plans and extended elevation, we would not only not be in the position to present our findings in a visual way, but the exploration itself would be much less focused. Certainly in the expeditions covered by this volume, the survey was used each year to zero in on potential areas for connecting \passage{Sistem Migovec} and \passage{Vrtnarija}, directly resulting in the exploration focus of 2008 and 2009 and several JSPDT 'super-actions'.



 		\end{multicols}
 	\end{fullwidth}
 \end{tcolorbox}


